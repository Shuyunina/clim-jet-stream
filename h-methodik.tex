
\chapter{Methodik}
Während der Arbeit an dieser Abschlussarbeit wurden zwei Verfahren zur Detektion von Jetstreams im Allgemeinen und zur Unterscheidung zwischen polarem und subtropischem Jetstream im Speziellen umgesetzt. Die erste Methode wurde eigens entwickelt und basiert auf einem Least-Squares-Fit mit Chebyshev-Polynomen. Im Folgenden wird diese Methode daher als \enquote{Chebyshev-Methode} bezeichnet. Die zweite Methode ist \citet{molnos-2017} entnommen und baut auf dem Dijkstra-Algorithmus auf, der kürzeste Pfade zwischen einem Start- und Zielknoten in Graphen ermittelt. Diese Methode wird im Folgenden \enquote{Dijkstra-Methode} genannt. 

\section{Untersuchte Datensätze}
In dieser Arbeit werden die ERA-Interim- \citep{dee-2011} mit den ERA-40-Daten \citep{uppala-2005} kombiniert, um einen konsistenten Datensatz von 1957 bis 2017 zu erhalten. Der ERA-Interim-Datensatz, der in 1979 beginnt, wird also mit dem ERA-40-Datensatz verlängert. Dieser Datensatz wird zunächst monatlich gemittelt und von der südlichen Hemisphäre bereinigt. Um im Anschluss an die Analyse der Reanalysedaten die Option zu haben, die Ergebnisse mit Klimaprojektionen zu vergleichen, werden die Daten vom vorliegenden $1,00^{\circ} \ast 1,00^{\circ}$-Gitter auf ein T63-Gitter extrapoliert.
\\
Betrachtet werden die Parameter Zonalwind $u$, Meridionalwind $v$ und geopotenzielle Höhe $z$ auf den Druckniveau $250\,$hPa, da \citet{rikus-2015} etwa in dieser Höhe den subtropischen und Polarfrontjetstream detektiert hat.
\\
Darüber hinaus wird der Sea-Ice-Index-Datensatz der NOAA \citep{sii-2016} genutzt, um Korrelationen zwischen der Position des Jets und der arktischen Eisfläche zu untersuchen.


\begin{table}
\caption[Eckdaten der genutzten Datensätze]{Genutzte Datensätze und deren Eckdaten wie zeitliche und räumliche Auflösungen und genutzte Parameter}
\centering
\begin{tabularx}{\textwidth}{|p{0.16\textwidth}|X|X|X|}
%{|>{\hsize=\hsize}X|>{\hsize=\hsize}X|>{\hsize=\hsize}X|>{\hsize=\hsize}X|}
%{|l|X|X|X|}
  \hline
  Datensatz & ERA-40 & ERA-Interim & NOAA-Sea-Ice \\
  \hline
  \multirow{2}{*}{}
    & \textbf{genutzt} & \textbf{genutzt} & \textbf{genutzt}  \\
    & (verfügbar) & (verfügbar) & (verfügbar) \\
  \hline 
  \multirow{2}{\linewidth}{Horizontaler Umfang}
    & \textbf{Nordhemisphäre} & \textbf{Nordhemisphäre} & \textbf{Summe NH} \\
    & (Global) & (Global) & (Globale Summe)  \\
    & & & \\
  \multirow{2}{\linewidth}{Horizontale Auflösung}
    & \textbf{T63-Gitter} & \textbf{T63-Gitter} & \textbf{Globale Summe} \\
    & ($1^{\circ}$-Gitter) & ($0,75^{\circ}$-Gitter) & (Globale Summe)  \\
    & & & \\
  \multirow{2}{\linewidth}{Vertikale Auflösung}
    & \textbf{250\,hPa-Level} & \textbf{250\,hPa-Level} & \textbf{Oberfläche} \\
    & (10 Drucklevel) & (20 Höhenlevel) & (Oberfläche)  \\
    & & & \\
  \multirow{2}{\linewidth}{Zeitlicher Umfang}
    & \textbf{Sep 1957 - Dez 1978} & \textbf{Jan 1979 - Jun 2017} & \textbf{Nov 1978 - Jun 2017} \\
    & (Sep 1957 - Dez 2002) & (Jan 1979 - Jun 2017) & (Nov 1978 - Jun 2017)  \\
    & & & \\
  \multirow{2}{*}{\parbox{0.16\textwidth}{Zeitliche Auflösung}}
    & \textbf{Monatliche Mittel der täglichen Mittelwerte} & \textbf{Monatliche Mittel der täglichen Mittelwerte} & \textbf{Monatliche Mittelwerte} \\
    & (00, 06, 12, 18 UTC) & (00, 06, 12, 18 UTC) & (Monatliche Mittel)  \\
    & & & \\
  \multirow{2}{*}{\parbox{0.16\textwidth}{Parameter}}
    & \textbf{Zonal- und Merdidionalwindgeschwindigkeit, geopotenzielle Höhe} & \textbf{Zonal- und Merdidionalwind, geopotenzielle Höhe} & \textbf{Seeeisausdehnung und -Fläche} \\
    & (00, 06, 12, 18 UTC) & (00, 06, 12, 18 UTC) & (Monatliche Mittel)  \\
  \hline
\end{tabularx}
\end{table}

\subsection{ERA-40}
\subsection{ERA-Interim}
\subsection{Sea-Ice-Daten}

\section{Chebyshev-Methode}
Der Chebyshev-Methode liegen folgende Annahme zugrunde:
\begin{enumerate}
  \item Die Positionen des Jetstreams bestimmen sich durch die Positionen lokaler meridionaler Maxima des Zonalwinds $u$.
  \item Falls mehrere lokale Maxima auffindbar sind, ist das nördlichere der beiden Maxima der Polarfront-Jetstream und das Südliche der Subtropen-Jetstream.
  \item Falls sich aufgefundene lokale Maxima nicht eineindeutig einem der beiden Jets zuordnen lassen oder die Distanz kleiner als $10^\circ$ ist, werden diese als Single-Jets klassifiziert.
\end{enumerate}

\subsection{Aufbau der Polynome und Least-Squares-Verfahren}
Die Maxima werden lokalisiert, indem ein Polynom mittels Least-Squares-Verfahren für jeden Meridian an den Zonalwind gefittet wird. Hierzu werden Chebyshev-Polynome erster Art genutzt. Diese haben den Vorteil, dass sich über die Koeffizienten nicht bloß der (geschätzte) Wert sondern auch die Werte der ersten und zweiten Ableitung zurückgegeben lässt. Über eine Nullstellensuche der ersten Ableitung und eine Überprüfung der zweiten Ableitung lassen sich so die Positionen und Werte der Maxima bestimmen. \\
Über die allgemeine Rekursionsformel für orthogonale Polynome \citep{Abramowitz-1972} lassen sie sich die Chebyshev-Polynome erster Art $T(x)$ und zweiter Art $U(x)$, die im Bereich $-1 < x < 1$ definiert sind, folgendermaßen berechnen:
\begin{align}
\begin{split}
& T_0(x) = 1, \: T_1(x) = x \\
& T_{n+1}(x) = 2 \cdot x \cdot T_n(x) - T_{n-1}(x)
\end{split}
\end{align} 
\begin{align}
\begin{split}
& U_0(x) = 1, \: U_1(x) = 2 \cdot x \\ 
& U_{n+1}(x) = 2 \cdot x \cdot U_n(x) - U_{n-1}(x)
\end{split}
\end{align}

Die Ableitungen lassen sich aus der trigenometrischen Definition der Polynome bestimmen und lauten nach \citet{spanier-1987} wie folgt:
\begin{align}
%\begin{split}
& \dfrac{d}{dx} T_n = \dfrac{n}{1-x^2} \left[ x \cdot T_n \left( x \right) - T_{n+1} \left( x \right) \right] = nU_{n-1} \left( x \right)
\end{align} 

\begin{align}
& \dfrac{d^2}{dx^2} T_n = 
  \begin{dcases} 
   \left(-1\right)^{n} \dfrac{n^4 - n^2}{3}  & \text{falls } x = -1 \\
   n \dfrac{(n+1) \cdot T_n - U_n}{x^2 - 1} & \text{falls } -1 < x < 1 \\
   \dfrac{n^4 - n^2}{3}                      & \text{falls } x = 1
  \end{dcases}
%\end{split}
\end{align} 


\subsection{Least-Squares-Fit}
Das Least-Squares-Verfahren oder die Methode kleinster Fehlerquadrate ist ein Verfahren der Regressionsanalyse, mit dem sich \enquote{die Form von Abhängigkeiten zwischen zwei  oder mehreren Merkmalen einer Grundgesamtheit untersuchen}~\citep{bronstein-2006} lässt. Im allgemeinen Fall wird so ein Zusammenhang zwischen einer unabhängigen Variable~$\underline{\boldsymbol{x}}$ und einer abhängigen Variable~$\underline{\boldsymbol{f}}$ gesucht. In diesem Fall sind dies der Breitengrad $\phi$ und die Intensität des Zonalwinds $u$. Über die vektoriellen Datenpunkte $\underline{\boldsymbol{f}}$ und die Schätzmatrix $\boldsymbol{G}$ lassen sich die Koeffizienten eines Polynoms $s$-ter Ordnung bestimmen.  
\begin{align}
\begin{split}
\underline{\boldsymbol{\tilde{a}}} = 
\begin{pmatrix}
  \tilde{a}_0 \\
  \tilde{a}_1 \\
  \vdots \\
  \tilde{a}_s
\end{pmatrix},
\underline{\boldsymbol{f}} = 
\begin{pmatrix}
  f_0 \\
  f_1 \\
  \vdots \\
  f_N
\end{pmatrix},
\boldsymbol{G} = 
 \begin{pmatrix}
  g_{0} \left( \underline{\boldsymbol{x}}^{\left( 1 \right)} \right) & g_{1} \left( \underline{\boldsymbol{x}}^{\left( 1 \right)} \right) & \cdots & g_{s} \left( \underline{\boldsymbol{x}}^{\left( 1 \right)} \right) \\
  g_{0} \left( \underline{\boldsymbol{x}}^{\left( 2 \right)} \right) & g_{1} \left( \underline{\boldsymbol{x}}^{\left( 2 \right)} \right) & \cdots & g_{s} \left( \underline{\boldsymbol{x}}^{\left( 2 \right)} \right) \\
  \vdots  & \vdots  & \ddots & \vdots  \\
  g_{0} \left( \underline{\boldsymbol{x}}^{\left( N \right)} \right) & g_{1} \left( \underline{\boldsymbol{x}}^{\left( N \right)} \right) & \cdots & g_{s} \left( \underline{\boldsymbol{x}}^{\left( N \right)} \right) \\
 \end{pmatrix}
\end{split}
\end{align}

\begin{align}
& \underline{\boldsymbol{\tilde{a}}} = \boldsymbol{G^TG^{-1}G^T}\underline{\boldsymbol{f}} 
\end{align}
Über die Koeffienten $\underline{\boldsymbol{\tilde{a}}}$ lassen sich einerseits die geschätzten Werte für den angenommenen Zusammenhang andererseits auch die Werte der ersten und zweiten Ableitung bestimmen. Die Schätzmatrizen $\boldsymbol{G}'$ und $\boldsymbol{G}''$ für die Ableitungen ergeben sich aus den Ableitungen der Chebyshev-Polynome erster Art.
\begin{align}
& \underline{\boldsymbol{\tilde{f}}} = \boldsymbol{G}\underline{\boldsymbol{\tilde{a}}}, \hspace{1em} \dfrac{d}{dx}\underline{\boldsymbol{\tilde{f}}} = \boldsymbol{G}'\underline{\boldsymbol{\tilde{a}}}, \hspace{1em} \dfrac{d^2}{dx^2}\underline{\boldsymbol{\tilde{f}}} = \boldsymbol{G}''\underline{\boldsymbol{\tilde{a}}}
\end{align}
Um die Maxima des Datensatzes zu finden, wird nun mithilfe des Newton-Raphson-Verfahrens nach Nullstellen der ersten Ableitung gesucht und überprüft, ob der Wert der zweiten Ableitung $<0$ ist. 
\subsection{Anwendung auf den Datensatz}
Auf den in dieser Arbeit genutzten Datensatz bezogen werden die Tropen ebenso wie der Pol vom Suchalgorithmus ausgeschlossen, um Fehler durch Randoszillationen und Ostwindmaxima in den Tropen auszuschließen. Konzentriert wird sich auf die Nordhemisphäre im Bereich $20^{\circ} < \phi < 85^{\circ}$.

\section{Dijkstra-Methode}
Die Dijkstra-Methode basiert auf der Methodik von \citet{molnos-2017}, der folgende Annahmen zugrundeliegen:
\begin{enumerate}
  \item Der Polarfront- und der Subtropen-Jetstream umlaufen die Erdkugel kontinuierlich.
  \item Die Positionen des Jetstreams bestimmen sich über einen kürzesten Pfad eines auf dem Längengrad-Breitengrad-Gitter des Datensatz definierten Graphen mit über gewichtete Kanten verbundene Knoten.
  \item Von jedem Knotenpunkt aus sind lediglich die drei bis acht horizontal, vertikal und diagonal benachbarten Knoten erreichbar
  \item Das Kantengewicht besteht aus den Parametern Windstärke, Windrichtung und Abweichung der Position vom klimatologischen Mittel. Die Gewichtung dieser Parameter findet über einen Vergleich mit einer anderen Methode statt und ist abhängig vom Typ des zu untersuchenden Jets sowie von der Jahreszeit
  \item Eine Unterscheidung zwischen subtropischem und polarem Jetstream wird durch eine Fallunterscheidung möglich gemacht, die verschiedene Parameter für die Position im klimatologischen Mittel annimmt.
  \item Falls der subtropische und der polare Jetstream näher als $10^\circ$ zusammenliegen, ist eine eindeutige Unterscheidung nicht möglich und der Mittelwert der beiden Positionen wird als Single-Jet klassifiziert.
\end{enumerate}
\subsection{Grundlegende Funktionsweise}
Das Problem des kürzesten Pfades in einem Graphen ist bereits vor einem halben Jahrhundert von \citet{kruskal-1956} und \citet{dijkstra-1959} untersucht worden. Der Fokus in dieser Arbeit liegt auf dem Dijkstra-Algorithmus, der das Problem für gegebene Startknoten und positiv definite Kantenlängen löst und so kürzeste Pfade zwischen einem Startknoten und allen übrigen Knoten findet. Ergänzend sei erwähnt, dass der Algorithmus auch für kürzeste Pfade zwischen allen Knoten und einem festgelegten Zielknoten verwendet werden kann, indem der Ziel- als Startknoten definiert und die Richtungen der Kanten umgekehrt werden. Im Folgenden wird zunächst die Funktionsweise des Dijkstra-Algorithmus im Allgemeinen und darauf aufbauend der Nutzen des Algorithmus für die Jetstream-Detektion erklärt.
\\
Gesucht ist der Pfad minimaler Länge zwischen einem beliebigen Startknoten $P$ und einem beliebigen Zielknoten $Q$. Angenommen der gesuchte Pfad führe über den Knoten $R$ und sei bekannt, dann ist auch der Pfad minimaler Länge von Startknoten $P$ zum Knoten $R$ bekannt. Dieses Wissen wird genutzt, um den Pfad minimaler Länge zu finden. Zunächst werden hierzu die Knoten und die Kanten in jeweils drei disjunkte Untermengen unterteilt:
\begin{itemize}
  \item Menge $A$: die Knoten, deren minimaler Pfad bekannt ist. Knoten werden dieser Menge mit ansteigender Entfernung vom Startknoten hinzugefügt;
  \item Menge $B$: die Knoten, die von den Knoten in Menge A direkt erreichbar sind, also deren Nachbarknoten;
  \item Menge $C$: die verbleibenden Knoten.
  \\
  \item Menge $I$: die Äste, die die Knoten in Menge A verbinden;
  \item Menge $II$: die Äste, die die Knoten in Menge mit den Knoten in Menge B verbinden;
  \item Menge $III$: die verbleidenden Äste.
\end{itemize}
Zunächst sind alle Knoten in Menge C und alle Äste in Menge III. Daraufhin wird der Startknoten P der Menge A hinzugefügt und die folgendenen zwei Schritte wiederholt ausgeführt, bis der Zielknoten Q erreicht ist:
\begin{enumerate}
  \item Zunächst werden alle Knoten, die den just zu Menge $A$ hinzugefügten Knoten mit Knoten der Untermengen $B$ und $C$ verbinden, als Nachbarknoten $R$ definiert. Falls sie vorher Teil der Untermenge $C$ waren, werden sie der Untermenge $B$ hinzugefügt. Daraufhin werden alle korrespondierenden Äste $r$ auf ihre Eigenschaften untersucht. Falls $R$ zu Menge $B$ gehört und der Ast $r$ einen kürzeren Pfad von $P$ nach $R$ als ein bereits Bekannter liefert, wird dieser Ast der Untermenge $II$ hinzugefügt und der längere Pfad wird verworfen. 
  \item Bei einer Beschränkung auf die Äste der Mengen $I$ und $II$ sind alle Knoten in Menge $B$ mit dem Startknoten $P$ verbunden. Demnach hat auch jeder Knoten in Menge $B$ einen Abstand von dem Knoten $P$. Der Knoten mit dem minimalen Abstand wird der Menge $A$ übertragen, der korrespondierende Ast der Menge $I$. Daraufhin wird Schritt 1 wiederholt, bis der Zielpunkt $Q$ der Menge $A$ zugeführt wird. 
\end{enumerate}
Der so gefundene Pfad ist der gesuchte Pfad minimaler Länge vom Startknoten~$P$ zum Zielknoten $Q$. \citep{dijkstra-1959}
\subsection{Praktische Anwendung}
Die größte Herausforderung in der Anwendung des Dijkstra-Algorithmus auf ein Problem wie die Detektion des Jetstreams ist die Definition der Knotenpunkte und der Kanten sowie deren Gewichte. Die Festlegung der Knotenpunkte ist relativ anspruchslos. Da der Datensatz auf einem Breiten-Längengrad-Gitter vorliegt, spiegelt jeder der Gitterpunkte einen Knotenpunkt wider. Um sicherzustellen, dass der 
Um einen vollständigen Umlauf des Jets zu berechnen, wird für jeden Zeitschritt die Datenreihe des ersten Längengrad-Schritts hinter die letzte Datenreihe gesetzt, sodass die erste Datenreihe gleich der letzten Datenreihe ist.
\\
Die Kanten führen von jedem Knoten lediglich zu seinen drei bis acht direkten Nachbarknoten, während sich die Gewichte $C_j$ aus den drei Komponenten Betrag der Windstärke $X_j$, Windrichtung $Y_j$ sowie der Abweichung vom klimatologischen Mittel $Z_j$ zusammensetzen (siehe Gleichung \ref{eq:dijkstra1}). Die Gewichtungen $w_1$, $w_2$ und $w_3$ der drei Beiträge sind positiv-semidifinit, summieren sich auf 1 und sind für jeden Beitrag spezifisch (siehe Gleichung \ref{eq:dijkstra2}).

\begin{align}
  & C_j = w_1 \cdot X_j + w_2 \cdot Y_j + w_3 \cdot Z_j   
  \label{eq:dijkstra1}\\
  & w_1 + w_2 + w_3 = 1
  \label{eq:dijkstra2}
\end{align} 

Da der Dijkstra-Algorithmus nur bei positiven Kantenlängen funktioniert, werden die verschiedenen Beiträge zuvor normiert. In Gleichung \ref{eq:dijkstra-x} wird der Betrag der Windgeschwindigkeit an Knoten $A$ und $B$, die über die Kante $j$ verbunden sind, aus den zonalen und meridionalen Anteilen $u_A$, $u_B$, $v_A$ und $v_B$ ermittelt und anhand der maximalen Windgeschwindigkeit im betrachteten Windfeld normiert.
\\
Die Windrichtung fließt über den zweiten Term $Y_j$ ein und ist in Gleichung \ref{eq:dijkstra-y} dargestellt. Hier sind $|\underline{\boldsymbol{V}}_A|$ die normierte Windrichtung an Knoten $A$ und $|\underline{\boldsymbol{e}}_j|$ der normierte Einheitsvektor in Kantenrichtung.
\\
Der dritte Beitrag $Z_j$ ist zur Differenzierung zwischen polarem und subtropischem Jetstream notwendig. Hierfür wird das Kantengewicht (Gleichung \ref{eq:dijkstra-z} abhängig von der Breitengradentfernung größer, sodass Positionen eher bevorzugt werden, die näher am klimatologischen Mittel sind. Durch den Einfluss der  vierten Potenz kann sich der Jet in einem Bereich von rund $\pm 20\,\%$ um das klimatologische Mittel relativ frei bewegen. Außerhalb dieses Bereichs steigt der Beitrag schnell gegen $1$.
\\
In der Praxis wird die minimale Pfadlänge für jeden einzelnen Breitengrad-Gitterpunkt bestimmt, das heißt bei dem genutzten Datensatz mit einer Längengrad-Breitengrad-Auflösung von $192 \ast 48$ werden $48$ unterschiedliche Pfade berechnet. Der gesuchte Pfad ist jener mit der kürzesten Weglänge.
\begin{align}
  & X_j = 1 - \dfrac{\sqrt{u_A^2 + v_A^2} + \sqrt{u_B^2 + v_B^2}}{2 \cdot \max_{k=1}^n  \sqrt{u_k^2 + v_k^2}}  
  \label{eq:dijkstra-x}\\
  & Y_j = \dfrac{1 - |\underline{\boldsymbol{V}}_A| \cdot |\underline{\boldsymbol{e}}_j|}{2} 
  \label{eq:dijkstra-y}\\
  & Z_J = \dfrac{\left( \phi_j - \phi_{clim} \right)^4}{\left[ \max \left( \phi_{clim}, 90 - \phi_{clim} \right) \right]^4}
  \label{eq:dijkstra-z}
\end{align}

Die Tabelle \ref{tab:dijkstra} zeigt die genutzten Parameter $w_1$, $w_2$, $w_3$ sowie $\phi_{clim}$ abhängig vom Typ des Jetstreams und der Jahreszeit. Diese Werte entstammen \citet{molnos-2017}, welche die Werte über eine Optimierung bestimmt haben. Für diese wurden die bestimmten Jetpositionen mit denen aus \citet{rikus-2015} verglichen. In \citet{rikus-2015} werden die Jetstreampositionen aus dem Längengradmittel des Zonalwindfeldes berechnet, also aus dem mittleren Luftdruck-Breitengrad-Feld vom Äquator zum Nordpol. Um diese Werte mit dem Dijkstra-Algorithmus, der zweidimensional auf dem Längengrad-Breitengrad-Feld arbeitet, zu vergleichen, werden die Positionen zunächst über alle Längengradschritte gemittelt. Die Differenz der so bestimmten Breitengradpositionen wird daraufhin über eine Variation der Parameter minimiert, woraus sich Tabelle \ref{tab:dijkstra} ergibt.

\begin{table}[htb]
  \caption[Parameter der Kantengewichtung der Dijkstra-Methode]{Optimierte Parameter der Kantengewichtung der Dijkstra-Methode nach \citet{molnos-2017}}
\centering
\begin{tabular}{|cl|rr|}
  \hline
  Jahreszeit & Parameter & Subtropenjet & Polarjet \\
  \hline
  \multirow{4}{*}{November - April} 
    & $w_1$ & $0,05$ & $0,05$ \\
    & $w_2$ & $-$ & $-$ \\
    & $w_3$ & $0,95$ & $0,95$ \\
    & $\phi_{clim}$ & $25,1^{\circ}$\,N & $67,5^{\circ}$\,N \\
  \hline
  \multirow{4}{*}{Mai - Oktober} 
    & $w_1$ & $0,07$ & $0,04$ \\
    & $w_2$ & $-$ & $-$ \\
    & $w_3$ & $0,93$ & $0,96$ \\
    & $\phi_{clim}$ & $29,8^{\circ}$\,N & $69,1^{\circ}$\,N \\
  \hline
\end{tabular}
\label{tab:dijkstra}
\end{table}

Obwohl sich der in dieser Arbeit genutzte Datensatz (2D, $250\,\text{hPa}$, Monatsmittel) von dem in \citet{molnos-2017} verwendeten Datensatz (2D, vertikal gemittelt über $500-150\,\text{hPa}$, 15-tägiges gleitendes Mittel) unterscheidet, wird keine erneute Optimierung der Parameter vorgenommen, insbesondere um die Dijkstra-Methode mit der Chebyshev-Methode anhand des selben Input-Datensatzes zu vergleichen. 

\section{Technische Umsetzung}
Für die programmiertechnische Umsetzung werden \texttt{R} und \texttt{python} und die Kommando\-zeilen\-werkzeuge \texttt{cdo} und \texttt{git} verwendet. Sie ist folgendermaßen erfolgt: 
\\
Zunächst werden die täglichen Werte verschiedener Parameter der Era-40- und Era-Interim-Datensätze als tägliche Werte mittels der Python-Library \texttt{ecmwfapi} \citep{ecmwfapi} heruntergeladen. Daraufhin werden diese Datensätze so bearbeitet, dass sich am Ende ein zusammenhängender Datensatz der monatlichen Mittelwerte des Zonalwinds, des Meridionalwinds und der geopotenziellen Höhe auf der Nordhemisphäre auf einem T63-Gitter im Zeitraum von 1957 bis 2016 ergibt. Dies wird mit dem Kommandozeilen-Werkzeug \texttt{cdo} \citep{cdo-2015} bewerkstelligt.
\\
Für die weitere Datenverarbeitung wird ausschließlich auf \texttt{R} \citep{R-2017} zurückgegriffen. Neben den für diese Arbeit geschriebenen Routinen und Paketen werden einige weitere Pakete genutzt. Dies sind im Wesentlichen \texttt{ncdf4} \citep{ncdf4-2017} für das Einlesen von Dateien im netCDF-Format, \texttt{chron} \citep{chron-2017} und \texttt{lubridate} \citep{lubridate-2011} für die Verarbeitung von Zeitstempeln, \texttt{rootSolve} \citep{rootSolve-2009} für das Lokalisieren von Nullstellen mit dem Newton-Raphson-Verfahren und \texttt{igraph} \citep{igraph-2006} zum Auffinden des kürzesten Weges zwischen Knoten in einem Graphen mit dem Dijkstra-Algorithmus.
Darüber hinaus werden \texttt{parallel} \citep{R-2017}, \texttt{foreach} \citep{foreach-2015} und \texttt{doParallel} \citep{doParallel-2015} für paralleles Rechnen sowie \texttt{reshape2} \citep{reshape2-2007} und \texttt{ggplot2} \citep{ggplot2-2009} zur Visualisierung der Ergebnisse genutzt.
\\
Bei Interesse ist der Code öffentlich unter \url{https://sebaki.github.io/clim-jet-stream/} einsehbar. Rückfragen sind per Mail an {\url{s.kiefer@posteo.de}} zu richten.
