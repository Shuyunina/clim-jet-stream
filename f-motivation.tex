\chapter{Motivation und Zielsetzung} \label{ch:motivation}

Der Klimawandel ist weltweit allgegenwärtig \citep{ipcc-2014}. Die Auswirkungen der globalen Erwärmung zeigen sich in schmelzenden Gletschern, steigenden Meeresspiegeln, Extremwetterereignissen wie Dürren und Überschwemmungen, im Monsun und in tropischen Zyklonen. Diese Auswirkungen und besonders die damit verbundenen Risiken, die sich aus der Vulnerabilitt und der Exposition gegenüber Gefahren durch natürliche Variabilität und anthropogene Klimaänderungen zusammensetzen, sind global ungleich verteilt. Beispielsweise sind ohnehin trockene Regionen wie der mittlere Osten für Dürren besonders anfällig \citep{ipcc-wg2-2014}. In Syrien ist eine mehrjährige Dürreperiode einer der Auslöser für den seit 2011 anhaltenden Bürgerkrieg und die Flucht der Menschen in den Libanon und nach Europa gewesen \citep{gleick-2014}.

Die Ursache für den globalen Temperaturanstieg sind in die Atmosphäre ausgestoßene Treibhausgase wie Kohlenstoffdioxid CO$_2$ und Methan CH$_4$. Der Klimawandel ist hauptsächlich menschengemacht \citep{ipcc-wg1-2013}. Seit 1850 wurden mehr als $1000$\,Gto\,CO$_2$ maßgeblich von den Industrienationen emittiert. Seitdem die Industrialisierung in den Schwellenländern Einzug gehalten hat, steigt deren Beitrag am globalen CO$_2$-Ausstoß insbesondere seit 2000 rasant an. Genannt seien hier insbesondere die BRICS-Staaten China und Indien, die in Absolutbeträgen die Liste der Emittenten bereits anführen, auf die Einwohnerzahl normiert jedoch deutlich hinter die Industriestaaten zurückfallen. \citep{ipcc-wg3-2014}

Dies offenbart eine enorme Diskrepanz, da die Industriestaaten, die historisch die Hauptemittenten von Treibhausgasen sind, die Auswirkungen des eigenen Handelns lediglich indirekt zu spüren bekommen. Auch wenn Auswirkungen spürbar sind wie in erhöhter Zuwanderung aus der MENA-Region in Süd- und Zentraleuropa, Sturmfluten in den Niederlanden oder tropischen Wirbelstürmen in den Vereinigten Staaten von Amerika, wird dies nur selten als Handlungsaufforderung zur Minderung der Treibhausgas-Emissionen begriffen. Statt die Wirtschaft auf eine nachhaltige Spur zu bringen, werden lediglich Anpassungs- und Abschottungsmaßnahmen ergriffen. In Deutschland rücken wirksame Maßnahmen zur Minderung in weite Ferne: die Energiewende stagniert, nach einer Wende im Verkehrssektor lässt sich lange suchen und am Paradigma des ewigwährenden Wirtschaftswachtums wird weiter festgehalten. 

Um zu ergründen, welcher Natur die großräumigen Auswirkungen des Klimawandels in den mittleren Breiten sind, wird in dieser Arbeit ein Wetterphänomen untersucht, dass für eben diese Regionen wetterbestimmend ist. Als eines, zu dem auch schon geforscht wird, sei an dieser Stelle der Polarfront-Jetstream benannt. Der Jetstream (oder auch Strahlstrom) wird von der Weltorganisation für Meteorologie definiert als ein starke schmale Strömung, die sich entlang einer horizontalen Achse durch eine starke horizontale und vertikale Windscherung auszeichnet \citep{wmo-1958}. Dieser mäandriert in Wellenform um den Nordpol, wobei die Welle selbst ostwärts propagiert. In den vergangenen Jahren hat es ein paar Male die Situation gegeben, dass der polare Jetstream über einen längeren Zeitraum seine Position nicht verändert hat, die Welle also nicht wie üblich propagierte. Diese Stagnation führte beispielsweise im Winter 2013/14 zu Rekordschneefällen an der Ostküste der Vereinigten Staaten von Amerika \citep{palmer-2014} und im Sommer 2003 zu einer Hitzewelle in Westeuropa \citep{petoukhov-2013}.

In den Klimawissenschaften herrscht Einigkeit darüber, dass ein Anstieg der globalen Temperatur mit hoher Wahrscheinlichkeit zu einem vermehrten Auftreten von Rekordtemperaturen führt. Ebenso erhöht sich der atmosphärische wassergehalt, was die Wahrscheinlichkeit von Starkregenereignissen erhöht. Die steigenden Temperaturen auf der Landfläche sorgen für eine höhere Verdunstung, was die Landwirtschaft änfälliger gegenüber Dürren macht. Während die Atmosphäre und der Ozean erwärmt wird, dehnt sich Meerwasser aus und Gletscher und Eisschilde schmelzen. Hierdurch steigt der Meeresspiegel. Verglichen mit diesen Wirkungsketten erscheint der hergestellte Zusammenhang zwischen einer Abnahme arktischen Seeeises und einem kälteren Winter in den Vereinigten Staaten zunächst kontraintuitiv. \citep{wallace-2014}

Über Änderungen in der globalen Zirkulation ließe sich auch in einem wärmeren Klima, das zwar grundsätzlich die Wahrscheinlichkeit für kalte Winter mindert, das häufigere Auftreten kalter Winter erklären. Klimatologische Veränderungen in der Häufigkeit unterschiedlicher Zirkulationsmuster manifestieren sich in einer Veränderung der Position der Jetstreams. Der polare Jetstream der nördlichen Hemisphäre führt Luftmassen über Rossby-Wellen von West nach Ost. Über diese Wellen werden kalte Luftmassen südwärts und umgekehrt warme Luftmassen nordwärts transportiert. So bekommen Regionen, die unter einem nach Süden führenden Jet liegen, wahrscheinlich kälteres Wetter zu spüren, während für Regionen, die unter einem nach Norden führenden Jet liegen, der umgekehrte Fall gilt. Sie erfahren wärmeres  Wetter. \citep{palmer-2014}

In dieser Arbeit werden zwei unabhängige Methoden zur Detektion von Jetstreams und zur Differenzierung zwischen Polarfront- und subtropischem Jetstream auf einen Datensatz angewandt, der aus ERA-Interim- und ERA-40-Daten von 1957 bis 2017 besteht. Die Positionen der Jetstreams sowie deren Zonal- und Meridionalwindgeschwindigkeiten werden daraufhin auf Veränderungen in Zeit und Raum untersucht. Dies geschieht sowohl auf jährlichen als auch auf saisonalen Skalen. Darüber hinaus wird der Zusammenhang zwischen der arktischen Amplifikation, der sich verstärkenden Erwärmung der Arktis, und den Veränderungen des Polarfrontjets analysiert.

\section*{Gliederung}
Gegliedert ist die Arbeit hierzu in folgende Abschnitte: Aufbauend auf der Motivation wird in Kapitel \ref{ch:hintergrund} der Stand der Forschung präsentiert. Darauffolgend (Kapitel \ref{ch:methodik}) werden die genutzten Datensätze sowie die beiden verwendeten Methoden zur Detektion von Jetstreams vorgestellt. In Kapitel \ref{ch:plausibilitaet} wird die Plausibilität der berechneten Jetstreampositionen zunächst anhand von zufällig ausgewählten Einzelfallbeispielen untersucht, bevor die beiden Methoden miteinander verglichen werden. Die zeitliche und raumzeitliche Analyse der Ergebnisse findet in Kapitel \ref{ch:klimatologie} statt, in dem Trends der Positionen und Windgeschwindigkeiten ebenso wie die Entwicklung der Variabilität untersucht werden. In Kapitel \ref{ch:seeeis} wird untersucht, ob ein Zusammenhang zwischen der Ausdehnung des arktischen Seeeises und der Position des Polarfrontjetstreams und dessen Variabilität besteht. Zum Schluss werden die Ergebnisse diskutiert.
