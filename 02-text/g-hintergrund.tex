\chapter{Stand der Forschung} \label{ch:hintergrund}

In diesem Kapitel wird der Stand der Forschung vorgestellt. Zu Veränderungen der großräumigen Zirkulation und insbesondere zu veränderten Wettermustern in den mittleren Breiten wird bereits geforscht. Daher wird an dieser Stelle eine knappe Historie wegweisender Forschungsarbeiten der vergangenen Jahre gegeben.

\citet{francis-2012} beschäftigen sich mit der Frage, wie die großräumige atmosphärische Zirkulation von der im Winterhalbjahr vom Ozean abgegebenen Wärme beeinflusst wird, die durch die Aufnahme solarer Strahlung von größer werdenden Freiwasserflächen zunimmt. Diese Fragestellung baut auf Arbeiten aus 2009 und 2010 auf, die herausgearbeitet haben, dass sich die bodennahen Luftschichten sowie der obere Teil der Arktis etwa doppelt so schnell erwärmt wie die gesamte nördliche Hemisphäre. Einzelne Extremwettereignisse haben zwar typischerweise einen dynamischen Ursprung, resultieren jedoch aus anhaltenden Zirkulationsmustern, die häufig mit blockierenden Rossbywellen mit hohen Amplituden verbunden sind. Als Beispiele hierfür werden die Hitzewellen 2010 in Europa und Russland, die Überschwemmungen am Mississippi 1993 und die Kälteereignisse in Florida im Winter 2010/11 genannt. 
Die Analyse unterstützt zwei Hypothesen, nach denen die arktische Amplifikation zu anhaltenden Zirkulationsmustern führen kann, die wiederum in Extremwetterereignissen resultieren können. Eine Ursache ist der schwächere meridionale Gradient der \SIrange{1000}{500}{\hecto\pascal}-Schichtdicke, was die Rossby-Welle auf eine nördlichere Amplitudenbahn drängt und so zu langsameren Zirkulationssystemen führt. Dies erhöht die Wahrscheinlichkeit für Extremwetterereignisse, die von persistenten Wetterlagen verursacht werden. Die zweite Ursache ist eine nordwärts gerichtete Verlagerung der Rücken der Wellen im \SI{500}{\hecto\pascal}-Druckniveau, was zu einer höheren Amplitude der Strömung führt und die Wahrscheinlichkeit von Blocking-Situationen weiter erhöht. Die Studie schließt damit, dass durch das weitere Schwinden der arktischen Meereisbedeckung davon auszugehen ist, dass die großräumige Zirkulation zunehmend durch die arktische Amplifikation beeinflusst wird.

Ziel von \citet{screen-2013} ist, die atmosphärischen Veränderungen besser zu verstehen, die möglicherweise als Reaktion auf die beobachteten Rückgänge des arktischen Meereises in den letzten drei Jahrzehnten eingetreten sind. Die Ergebnisse aus Simulationen mit zwei unabhängigen allgemeinen generellen Zirkulationsmodellen, deren einziger Antrieb die Variationen der Beobachtungen des arktischen Meereises zwischen 1979 und 2009 waren, deuten darauf hin, dass atmosphärische Auswirkungen auf den Verlust des Meereises am stärksten innerhalb der maritimen und Küstenregionen der Arktis auftreten. Die Modelle legen nahe, dass die Abnahme des arktischen Meereises den Energietransfer vom Ozean in die Atmosphäre, die verstärkte Erwärmung und die zunehmende Feuchte der unteren Troposphäre, die Stärke der Inversion sowie die erhöhte Schichtdicke der unteren Troposphäre beeinflusst. Die Modelle zeigen kaum Anzeichen für eine durch das Meereis hervorgerufene Temperaturänderung oberhalb der stabilen Grenzschicht, obwohl Beobachtungen und Analysen bereits auf eine Temperaturerhöhung hindeuten. Die beobachtete arktische Erwärmung in der Höhe wird wahrscheinlich durch Änderungen der Oberflächentemperatur des Meeres und die resultierende Zunahme des polumschlagenden Wärmetransports in die Arktis verursacht. Die Modellexperimente ermöglichen die Isolierung von durch Meereis verursachten atmosphärischen Veränderungen, die sich sowohl in gekoppelten Modellversuchen als auch in der Natur nur schwer entschlüsseln lassen. Die Analyse verschiedener Kälteperioden der vergangenen Jahre hat gezeigt, dass die Modelle keine robuste und weit verbreitete Abkühlung und keine erhöhten Schneefälle als Antwort auf die arktische Amplifikation in den letzten drei Jahrzehnten simulieren. Dies schließt zwar nicht aus, dass in den Wintern 2009/10 und 2010/11 die Meereisbedingungen eine treibende Rolle spielten, aber es deutet darauf hin, dass die vorgeschlagenen Verbindungen zwischen multidekadalen Änderungen des Meereises und der borealen Winterkühlung möglicherweise verfrüht sind.

\citet{petoukhov-2013} haben sich mit der quasiresonanten Amplifikation von planetaren Wellen und Wetterextrema der nördlichen Hemisphären auseinandergesetzt. Im Nachgang des außergewöhnlichen Sommers in 2003 in Europa wurde vorgeschlagen, dass der beobachtete Klimatrend die Wahrscheinlichkeitsverteilung der Sommertemperaturen in Richtung wärmerer Werte verschiebt und diese Verteilung verbreitert, sodass Extremwerte wahrscheinlicher werden. Jedoch erklärt auch eine veränderte Wahrscheinlichkeitsverteilung die regionalen Sommerextrema der vergangenen Jahre nicht vollständig, sodass angenommen wird, dass kein rein stochastischer Extremwirkungsmechanismus am Werk ist. Die Hitzewellen in Russland in 2010 sowie in den Vereinigten Staaten in 2011, deren Muster sich räumlich über die gesamte Hemisphäre und zeitlich über den gesamten Sommer erstreckten, legen nahe, dass es sich nicht um typische Blocking-Lagen mit einer charakteristischen Zeit von fünf bis sieben Tagen handelt. Eine quasistationäre freie synoptisch-skalige Welle schwingt quasiresonant, sobald die beiden so genannten Wendepunkte der mittleren Breiten für die Welle auftreten, sodass eine starke dynamische Reaktion auf die klimatologischen diabatischen und orographischen Antriebe begünstigt wird. Die vorliegenden Daten und Ergebnisse deuten auf eine Veränderung der atmosphärischen Bedingungen hin, sodass die betrachtete quasiresonante Amplifikation häufiger auftreten könnte.

In \citet{barnes-2013-b} wird im CMIP5-Modell (Coupled-Model-Inter\-com\-parison-Project Phase 5) untersucht, wie Jetstreams über den mittleren Breiten auf eine Zunahme der CO$_2$-Konzentration in der Atmosphäre reagieren. In allen untersuchten Regionen (Nordatlantik, Nordpazifik und südliche Hemisphäre) zeigen die Analysen eine Verschiebung der Jetstreams in Richtung der Pole um etwa \SIrange{1}{2}{\degree} zum Ende des 21. Jahrhunderts. Darüber hinaus verändert sich die Variabilität. So wird für den Nordatlantik ein eher pulsierender und weniger mäandrierender Jet prognostiziert, während für den Nordpazifik der umgekehrte Fall gilt. Da einige Projektionen starke Veränderungen in der Jetposition simulieren, halten \citet{barnes-2013-b} es für möglich, dass sich das Variabilitätsmuster in Zukunft deutlich vom bekannten Verhalten der Jetstreams unterscheidet. Die Jetstreamvaribilität ist stark verbunden mit den Zugbahnen von Tiefdruckgebieten, dem regionalem Wetter sowie mit Blockings von Hochdruckgebieten. Durch diese vielfältigen Verbindungen zu physikalischen Prozessen in der Troposphäre und der Erdoberfläche werden Veränderungen in den dominanten Variabilitätsformen der Jetstreams wichtige globale Auswirkungen haben.

\citet{barnes-2013-a} quantifiziert beobachtete Trends der meridionalen Ausdehnung von Wellen über dem Nordatlantik und -amerika mit Hilfe von zwei verschiedenen Maßen und drei Reanalysen. Die Maße zeigen nicht übereinstimmend, ob ein signifikanter Trend der Veränderungen der Wellenamplituden beobachtet wird. Die Uneinigkeit entstammt der Autorin zufolge den unterschiedlichen Methoden, die Wellen entweder auf täglichen oder auf saisonalen Zeitskalen definieren. Verlangsamt haben sich großskalige Wellen in den vergangenen Jahrzenten lediglich in den Herbstmonaten signifikant, wobei die Signifikanz des Trends sensibel auf Änderungen der Region reagiert. Darüber hinaus ist zu keiner Jahreszeit ein signifikanter Anstieg des Auftretens von Blockings festzustellen. Sie kommt zu dem Schluss, dass die Hypothese, dass eine verstärkte polare Erwärmung zu einem vermehrten Auftreten von sich langsam verlagernden Wettermustern führt, nicht durch Beobachtungen gestützt wird. Die Arktis verändere sich rasant und diese Veränderungen werden voraussichtlich tiefgreifende Auswirkungen auf die nördliche Hemisphäre haben. Diese Studie zeigt jedoch, dass die Beziehung zwischen der arktischen Amplifikation und dem Wetter der mittleren Breiten komplex ist. Zusätzliche Einflüsse aus anderen Breitengraden sowie interne Variabilität spielen wahrscheinlich eine wichtige Rolle bei der Bestimmung der atmosphärischen Trends und gezielte Modellierungsstudien seien notwendig, um die relative Bedeutung der Polarveränderungen auf das atlantische Wetter zu quantifizieren.

Aufbauend auf Arbeiten aus 2012 und 2013, die zeigen, dass die Erwärmung der Arktis und die Abnahme von polarem Meereis zu einer Zunahme von Blockings auf der nördlichen Hemisphäre geführt haben, untersuchen \citet{barnes-2014}, ob die Häufigkeit solcher Blockings in den letzten Jahrzehnten robuste Trends aufweisen. Dies wird mit wird mit drei Methoden in vier Reanalysen untersucht. Für den Blocking-Index ist zu keiner Jahreszeit eine deutliche Zunahme von Blockings festzustellen, obwohl einzelne Regionen und Jahreszeiten robuste Unterschiede zeigen. Vergleiche der Blockingfrequenzen im September zwischen Jahren mit einer starken Ausdehnung des Seeeises und jenen mit einer geringen Seeeisausdehnung zeigen deutliche Unterschiede. In den Sommermonaten zeigen sich positive Differenzen über dem Nordatlantik und Negative über dem Nordpazifik. Es wird davor gewarnt, die Ergebnisse schlicht als Nachweis von Blockings als Antwort auf Seeeisverluste zu interpretieren. Die Veränderungen können vielmehr von einer Vielzahl unterschiedlicher dynamischer Mechanismen ausgelöst werden. Insgesamt stützen diese Schlussfolgerungen die Untersuchungen von \citet{barnes-2013-a}, dass der Zusammenhang zwischen der arktischen Amplifikation und häufigeren Blockadesituationen auf der nördlichen Hemisphäre derzeit nicht durch Beobachtungen gestützt wird. Während das arktische Meereis in den vergangenen Jahren nie da gewesene Verluste verzeichnet, scheinen Blockings nicht außergewöhnlich und liegen innerhalb der historisch beobachteten Größenordnung. Die große Variabilität der Blocking-Ereignisse, sowohl auf interannualer als auch auf dekadischer Zeitskala, zeigt die Schwierigkeit, eine potenziell durch den Klimawandel induzierte Reaktion von der natürlichen Variabilität zu unterscheiden.

\citet{cohen-2014} haben sich in einem Review-Artikel mit der arktischen Amplifikation und Extremwetter in den mittleren Breiten beschäftigt. Die rapide Erwärmung der Arktis hat zu einem dramatischen Abschmelzen des arktischen Meereises und der Schneedecke im Frühjahr geführt, und zwar in einem höheren als durch Klimamodelle simulierten Tempo. Diese tiefgreifenden Veränderungen des arktischen Systems sind mit einer Periode von vermeintlich häufiger auftretenden extremen Wetterereignissen in den mittleren Breitengraden der nördlichen Hemisphäre zusammengefallen. Die Möglichkeit eines Zusammenhangs zwischen dem arktischem Wandel und dem Wetter der mittlerem Breiten hat Forschungsaktivitäten angestoßen, die drei mögliche dynamische Wege aufzeigen, die die arktische Amplifikation mit den mittleren Breiten verbinden: Veränderungen der Zugbahnen von Tiefdruckgebieten, des Jetstreams und der planetaren Wellen. Durch die Veränderung dieser atmosphärischen Schlüsselmerkmale ist es prinzipiell möglich, dass Meereis und Schneedecke gemeinsam das Wetter in den mittleren Breiten beeinflussen. Aufgrund unvollständiger Kenntnisse über den Einfluss des Klimawandels auf diese Phänomene, kombiniert mit spärlichen und kurzen Datensätzen und unvollkommenen Modellen, bleiben jedoch große Unsicherheiten über die Größenordnung eines solchen Einflusses bestehen. \citet{cohen-2014} kommen zu dem Schluss, dass ein verbessertes Prozessverständnis, anhaltende und zusätzliche arktische Beobachtungen und besser koordinierte Modellierungsstudien erforderlich sind, um das Verständnis für Einflüsse auf das Wetter in den mittleren Breiten und dortige Extremereignisse zu verbessern.
Ein fundamentaler Treiber des Polarfrontjets ist der Temperaturunterschied zwischen der Arktis und den mittleren Breitengraden. Daher könnte eine niedrigere Temperaturdifferenz zu einer schwächeren Zonalwindkomponente und größeren Mäandern führen. Eine schwächere und mäandrierende Strömung kann dazu führen, dass die Wettersysteme langsamer nach Osten wandern, sodass persistente Wettermuster häufiger werden. Darüber hinaus führt die arktische Verstärkung dazu, dass die Schichtdicke der Atmosphäre nach Norden zunimmt, sodass sich die atmosphärischen Rücken nach Norden verlängern und somit die Amplituden der Strömung vergrößern können. Wetterextreme treten häufig auf, wenn die atmosphärischen Zirkulationsmuster anhaltend sind, was bei einer starken meridionalen Windkomponente tendenziell der Fall ist.
Einige Aspekte dieser hypothetischen Verknüpfung werden durch Beobachtungen und Modellsimulationen unterstützt. Ein signifikanter Rückgang des mittleren Zonalwinds bei \SI{500}{\hecto\pascal} im Herbst wird regional beobachtet. Dies lässt sich an der thermischen Windbeziehung ablesen, die besagt, dass die vertikale Windscherung proportional zum meridionalen Temperaturgradienten ist. Unter der Annahme, dass die Winde an der Oberfläche nicht zunehmen, sollte der zonale Wind auf Jetstream-Niveau mit einem schwächeren meridionalen Temperaturgefälle nachlassen. In anderen Jahreszeiten, in denen die arktische Amplifikation schwächer ist, ist kein signifikanter Trend des mittleren Zonalwinds zu beobachten.
Die Herausforderung bestehe jedoch weiterhin darin, die arktische Amplifikation direkt mit Änderungen der Geschwindigkeit und Struktur des Strahlstroms zu verknüpfen. Beispielsweise beeinflussen neben dem oberflächennahen meridionalen Temperaturgradienten weitere Faktoren den Jetstream, darunter die Rückkopplungen von synoptischen Eddies und Stürmen sowie der meridionale Temperaturgradient der oberen Tro\-po\-sphä\-re. Obwohl die arktische Verstärkung den oberflächennahen meridionalen Temperaturgradienten geschwächt hat, hat sich der Temperaturgradient zwischen den Tropen und den mittleren Breiten in der Höhe verstärkt, was zu einem Anstieg der Winde auf Jetstream-Niveau führen würde. Eine weitere Herausforderung besteht darin, herauszufinden, ob und wie groß der Anteil entfernter Veränderungen auf die arktische Amplifikation ist. Diese Unterscheidung ist von großer Bedeutung, denn wenn ein erheblicher Anteil aus der Ferne getrieben wird, kann die arktische Amplifikation teilweise eher als Reaktion auf das Wetter der mittleren Breiten und weniger als dessen Antrieb betrachtet werden.

Nach \citet{francis-2015} verschärft sich der Klimawandel in der Arktis weiter, das Schwinden des arktischen Meereises hält an und die Masse des grönländischen Inlandeises nimmt ab, die Schneebedeckung der nördlichen Hemisphäre während des Frühsommers geht zurück und die arktische Amplifikation hält weiter an. Der überproportionale Temperaturanstieg beeinflusst die großräumige Zirkulation möglicherweise mit weitreichenden Auswirkungen. Der NOAA-tabulated-climate-extreme-index \citep{karl-1996} hat in den Vereinigten Staaten im Vergleich mit den Jahren vor der arktischen Amplifikation um etwa ein Drittel zugenommen. Noch ist unklar, ob die Erwärmung der Arktis hierfür eine Ursache ist. \citet{francis-2015} weisen nach, dass in Regionen und Jahreszeiten, in denen sich die meridionalen Gradienten als Reaktion auf die Arktis abgeschwächt haben, der Fluss der oberen Troposphäre meridionaler oder welliger geworden ist. Darüber hinaus hat in den letzten Jahren die Häufigkeit von Tagen mit  Jetstreams mit hoher Amplitude zugenommen. Diese Muster, die sich durch hohe Amplituden auszeichnen, sind dafür bekannt, dass sie anhaltende Zirkulationsmuster erzeugen, die zu extremen Wetterereignissen führen können. Als Beispiele werden die kalten, schneereichen Winter im Osten der Vereinigten Staaten in den Wintern 2009/10, 2010/11 und 2013/14, die Rekordschneefälle in Japan und im Südosten Alaskas im Winter 2011/12 sowie die Überschwemmungen im Mittleren Osten im Winter 2012/2013 angeführt. Auf der Grundlage dieser Ergebnisse kommen sie zu dem Schluss, dass ein Fortschreiten der arktischen Amplifikation zu allen Jahreszeiten infolge des ungebremsten Anstiegs der Treibhausgasemissionen zu einem zunehmend welligen Charakter der Jetstreams der hohen Troposphäre und damit zu einer Zunahme extremer Wetterereignisse beitragen wird.

\citet{rikus-2015} hat ein einfaches Detektionsschema über geschlossene Konturlinien für den meridional und monatlich gemittelten Zonalwind entwickelt, um Jetstreams zu identifizieren, und auf \num{9} Reanalysedatensätze von \numrange{1979}{2009} angewandt. Um vermeintliche Jets zu lokalisieren, wird das über alle Meridiane und monatlich gemittelte Zonalwindfeld als zweidimensionales Bild betrachtet. Auf dieses wird jeweils der maximum-Filter und der minimum-Filter des Moduls \texttt{ndimage} aus dem \texttt{Python}-Paket \texttt{SciPy} auf einen Radius von zwei Punkten in beiden Richtungen angewandt, um ein Maxima- und ein Minima-Bild zu generieren. Lokale Maxima sind dann jene Punkte, deren Werte im Original und im mittels maximum-Filter bearbeiteten Bild gleich sind und deren Differenzen zwischen mit minimum- und maximum-Filter prozessierten Bildern kleiner als ein Schwellwert von \SI{0.4}{\metre\per\second} sind. Schwellwert und Radius werden manuell getestet und so optimiert. Er findet sechs verschiedene Jets, von denen jene über den mittleren Breiten in der oberen Troposphäre näher betrachtet werden. Der Vergleich der Datensätze zeigt substantielle Übereinstimmung der gefundenen Positionen in allen Datensätzen. Gefunden wird unter anderem ein Troposphärenjet auf der Nordhemisphäre, der zwischen \SI{20}{\degree} und \SI{54}{\degree} nördlicher Breite und in einer Höhe zwischen \SI{400}{\hecto\pascal} und \SI{110}{\hecto\pascal} detektiert wird. Der arktische Jetstream liegt zwischen \SI{55}{\degree} und \SI{90}{\degree} nördlicher Breite und in einer Höhe zwischen \SI{500}{\hecto\pascal} und \SI{210}{\hecto\pascal}.

\citet{coumou-2015} haben am Thema der schächer werdenden Zirkulation in den mittleren Breiten der Nordhemisphäre während der Sommermonate gearbeitet. Die stärksten Änderungen seien in Herbst und Winter zu erwarten, in dieser Arbeit berichten sie jedoch über eine signifikante Abschwächung der Zirkulation in drei Schlüsselindikatoren: der Eddy-kinetischen Energie, der gemittelten Zonal\-wind\-ge\-schwin\-dig\-keit sowie der Amplitude schneller Rossbywellen. Allgemein zeichnen sich die großräumigen Atmosphärendynamiken der mittleren Breiten durch schnell wandernde freie (transiente) Rossbywellen mit Wellenzahlen $>6$ und durch Rossbywellen mit kleineren Wellenzahlen aus, die aufgrund von thermischen und orographischen Zwängen quasistationär liegen. Der Fokus der Arbeit liegt auf transienten Rossbywellen, die mit synoptischen Zyklonen und Antizyklonen assoziiert werden. Sie haben eine relativ schnelle Phasengeschwindigkeit und verursachen Wettervariabilität auf Zeitskalen von weniger als einer Woche. Typischerweise werde die Intensität der synoptischen Wellenaktivität durch die Anwendung eines Bandpassfilters auf hochaufgelöste Windfelddaten geschätzt, die Eddykinetische Energie (EKE) wird extrahiert. Diese ist ein Maß für das Zusammenspiel von Intensität und Frequenz von Hoch- und Tiefdrucksystemen. Quasistationäre Wellen werden durch die niedrigeren Frequenzen aus den EKE-Berechnungen ausgeschlossen. Im Zeitraum 1979 bis 2013 ist die EKE während der Sommermonate kontinuierlich zurückgegangen. Diese Beobachtung ist auf allen Druckebenen und über alle relevanten Breitengrade hinweg signifikant, wobei die stärksten Änderungen in der unteren bis mittleren Troposphäre erkannt werden. Die Veränderungen seien daher nicht auf eine Nord-Süd-Verschiebung der Strukturen zurückzuführen sondern vielmehr eine räumlich homogene Schwächung. Für die anderen Jahreszeiten sind die Trends der EKE ebenfalls rückläufig aber nicht signifikant. Der Rückgang der EKE im Sommer geht einher mit einem Rückgang des mittleren Zonalwinds. Diese Schwächung ist wiederum in allen Druckniveaus und verschiedenen Reanalysen zu beobachten. Die langfristige Schwächung der zonalen Strömung steht im Einklang mit dem Rückgang des thermischen Gradienten vom Nordpol zu den mittleren Breiten. Während der relative Rückgang der EKE in den zwischen 1979 bis 2013 zwischen \SI{8}{\percent} und \SI{15}{\percent} beträgt, schwächt sich der Zonalwind lediglich um \SIrange{4}{6}{\percent}. Ein ähnliches Verhältnis von den Veränderungen der EKE zu jenen der Zonalströmung zeigt sich auch in den Klimaprojektionen der CMIP5-Modelle. 

\citet{dicapua-2016} beschäftigen sich auf der Grundlage der statistischen Koinzidenz von hochamplitudenreichen Rossbywellen in der Strömung mittlerer Breitengrade und Wetterextremen an der Erdoberfläche mit der Entwicklung eines Mäander-Index, der die Welligkeit der mittleren Troposphäre misst. In anderen Studien wurden verschiedene Mechanismen vorgeschlagen, wie die Wirkungsweise der Rossbywellen aussehen kann. So können anormale Temperaturen der Meeresoberfläche im tropischen Pazifik quasistationäre Wellen erzeugen, die den Jetstream über den mittleren Breiten stören und verändern können. \citet{trenberth-2014} zeigen, dass einige der beobachteten  Zirkulationsanomalien in den letzten Jahren möglicherweise aus einer negativen Phase der Pacific Decadal Oscillation und den damit verbundenen tropischen Niederschlagsanomalien hervorgegangen sind, wobei die stärksten Veränderungen im Winter zu verzeichnen sind. Der Mäander-Index repräsentiert mögliche Positionsveränderungen der Wellenaktivität durch eine Suche nach dem Maximum der Welligkeit zu einem bestimmten Zeitpunkt. Der Mäanderindex erfasst die Gesamtwelligkeit in der Atmosphäre, die auch bei hohen Wellenzahlen ausgeprägt sein kann. Unabhängig von den zugrunde liegenden Faktoren könnte die Tendenz zu quasistationären Wellen während des borealen Sommerhalbjahres zu einer höheren Persistenz der Witterung und längeren Hitze- und Dürreperioden beitragen. Ziel von \citet{dicapua-2016} ist es, die maximale Welligkeit in der Atmosphäre der nördlichen Hemisphären über einen Index zu analysieren, der nicht durch Bewegungen der Welle in vertikaler oder meridionaler Richtung beeinflusst wird. Es werden einige robuste und signifikante Veränderungen mit diesem Index erkannt, die für Extreme im Bereich der mittleren Breiten relevant und die mit stark mäandernden Strömungsmustern verbunden sind.

\citet{kornhuber-2016} haben Wellenresonanzen als Hauptmechanismus für die Entstehung von quasistationären Wellen mit hohen Amplituden während des borealen Sommers untersucht. Sie haben die Bedingungen für die quasiresonante Amplifikation (QRA) in ein Detektionsschema einfließen lassen, um mittels einer objektiven Methode große Datenmengen nach QRA-Ereignissen zu durchsuchen. Die Untersuchung von Reanalysedaten von 1979 bis 2015 zeigt, dass bei etwa einem Drittel aller Ereignisse mit hoher Amplitude und Wellenzahlen \numlist{6; 7; 8} die QRA-Bedingungen erfüllt werden. Es wird verglichen mit der Klimatologie ein erhöhter Anteil von langsam laufenden Wellen mit hoher Amplitude gezeigt. Dieser Anstieg ist für Wellenzahlen \numlist{6; 7} signifikant, für die Wellenzahl \num{8} ist der Trend auch vorhanden, die Änderung ist jedoch nicht signifikant. Der Zeitpunkt der QRA-Detektion geht der maximalen Wellenamplitude etwa eine Woche voraus. Viele der detektierten Ereignisse gehen mit Doppeljetstrukturen im über alle Meridiane gemittelten Zonalwind einher. Die Ergebnisse der Analyse deuten auf eine Korrelation zwischen Extremwetterereignissen am Boden und stationären Wellen mit hohen Amplituden über den mittleren Breiten hin. Solche Extremereignisse treten häufig in den Regionen starker Meridionalwinde auf. Resonanzen allein führen jedoch nicht unbedingt zu Extremen, sondern schaffen günstige Bedingungen für das Auftreten dieser. So erhöhen quasiresonante Amplifikationen die Wahrscheinlichkeit für extremes Wetter.

\citet{molnos-2017} haben eine objektive Methode entwickelt, Jetstreamkerne zu identifizieren. Die Methode basiert auf dem Dijkstra-Algorithmus, der in Graphen kürzeste Pfade zwischen Knoten findet. Optimiert wurde die Methode an den von \citet{rikus-2015} gefundenen Positionen, angewandt auf Reanalysedaten von \num{1979} bis \num{2015}, die massengewichtet für die elf Druckniveaus zwischen \SI{500}{\hecto\pascal} und \SI{150}{\hecto\pascal} und gleitend für \num{15} Tage gemittelt werden. Die Analysen der Wahrscheinlichkeitsdichten zeigen, dass in den Wintermonaten zwei getrennte Jetstreams über Europa und Asien (\SI{20}{\degree} westlicher Länge bis \SI{140}{\degree} östlicher Länge) liegen, während die Jetstrukturen in den Sommermonaten eher einen verbundenen Jetstream über der westlichen Hemisphäre und sowohl getrennte als auch verbundene Jets über der östlichen Hemisphäre aufweisen.
