\chapter{Motivation und Zielsetzung} \label{ch:motivation}

Der Klimawandel ist weltweit allgegenwärtig \citep{ipcc-2014}. Die Auswirkungen der globalen Erwärmung zeigen sich in schmelzenden Gletschern, steigenden Meeresspiegeln, Extremwetterereignissen wie Dürren und Überschwemmungen, im Monsun und in tropischen Zyklonen. Diese Auswirkungen und besonders die damit verbundenen Risiken, die sich aus der Vulnerabilität und der Exposition gegenüber Gefahren durch natürliche Variabilität und anthropogene Klimaänderungen zusammensetzen, sind global ungleich verteilt. Beispielsweise sind ohnehin trockene Regionen wie der mittlere Osten für Dürren besonders anfällig \citep{ipcc-wg2-2014}. In Syrien ist eine mehrjährige Dürreperiode einer der Auslöser für den seit 2011 anhaltenden Bürgerkrieg und die Flucht der Menschen in den Libanon und nach Europa gewesen \citep{gleick-2014}.

Die Ursache für den globalen Temperaturanstieg sind in die Atmosphäre ausgestoßene Treibhausgase wie Kohlenstoffdioxid CO$_2$ und Methan CH$_4$. In 2016 erreichte der CO$_2$-Gehalt der Atmosphäre mit \num{403.3}\,ppm einen neuen Höchststand, der zuletzt im mittleren Pliozän vor 3 bis 5 Millionen Jahren erreicht wurde. Auch die Änderungsrate des Kohlenstoffdioxidgehalts in den vergangenen \num{70}\,Jahren hat sich verglichen mit dem Ende der letzten Eiszeit verhundertfacht. \citep{wmo-ghg-2017} 

Der Klimawandel ist hauptsächlich menschengemacht \citep{ipcc-wg1-2013}. Seit 1850 wurden mehr als \SI{1000}{\giga\tonne}\,CO$_2$ maßgeblich von den Industrienationen emittiert. Seitdem die Industrialisierung in den Schwellenländern Einzug gehalten hat, steigt deren Beitrag am globalen CO$_2$-Ausstoß insbesondere seit 2000 rasant an. Genannt seien hier insbesondere die BRICS-Staaten China und Indien, die in Absolutbeträgen die Liste der Emittenten bereits anführen, auf die Einwohnerzahl normiert jedoch deutlich hinter die Industriestaaten zurückfallen. \citep{ipcc-wg3-2014}

Dies offenbart eine enorme Diskrepanz, da die Industriestaaten, die historisch die Hauptemittenten von Treibhausgasen sind, die Auswirkungen des eigenen Handelns lediglich indirekt zu spüren bekommen. Auch wenn Auswirkungen spürbar sind wie in erhöhter Zuwanderung aus der MENA-Region in Süd- und Zentraleuropa, Sturmfluten in den Niederlanden oder tropischen Wirbelstürmen in den Vereinigten Staaten von Amerika, wird dies nur selten als Handlungsaufforderung zur Minderung der Treibhausgas\-emissionen begriffen. %Statt die Wirtschaft auf eine nachhaltige Spur zu bringen, werden lediglich Anpassungs- und Abschottungsmaßnahmen ergriffen. In Deutschland rücken wirksame Maßnahmen zur Minderung in weite Ferne: die Energiewende stagniert, nach einer Wende im Verkehrssektor lässt sich lange suchen und am Paradigma des ewigwährenden Wirtschaftswachtums wird weiter festgehalten. \citep{atlas-2014}

Um zu ergründen, welcher Natur die großräumigen Auswirkungen des Klimawandels in den mittleren Breiten sind, wird in dieser Arbeit ein Wetterphänomen untersucht, das für eben diese Regionen wetterbestimmend ist. Als eines, zu dem bereits geforscht wird, sei an dieser Stelle der Polarfront-Jetstream genannt. Der Jetstream wird von der Weltorganisation für Meteorologie definiert als eine starke schmale Strömung, die sich entlang einer horizontalen Achse durch eine starke horizontale und vertikale Windscherung auszeichnet \citep{wmo-1958}. Dieser mäandriert in Wellenform um den Nordpol, wobei die Welle abhänig von der Wellenlänge und dem Antrieb eher stationär liegt oder relativ zur Erdoberfläche selbst ostwärts propagiert. In den vergangenen Jahren hat es ein paar Male die Situation gegeben, dass der polare Jetstream über einen längeren Zeitraum seine Position nicht verändert hat, die Welle also nicht wie üblich propagierte. Diese Stagnation führte beispielsweise im Winter 2013/14 zu Rekordschneefällen an der Ostküste der Vereinigten Staaten von Amerika \citep{palmer-2014} und im Sommer 2003 zu einer Hitzewelle in Westeuropa \citep{petoukhov-2013}.

In den Klimawissenschaften herrscht Einigkeit darüber, dass ein Anstieg der globalen Temperatur mit hoher Wahrscheinlichkeit zu einem vermehrten Auftreten von Rekordtemperaturen führt. Ebenso erhöht sich der atmosphärische Wasserdampf, was die Wahrscheinlichkeit für Starkregen erhöht. Die steigenden Temperaturen an Land sorgen für eine höhere Verdunstung, was die Risiken für die Landwirtschaft gegenüber Dürren erhöht. Während die Atmosphäre und der Ozean erwärmt werden, dehnt sich Meerwasser aus und Gletscher und Eisschilde schmelzen. Hierdurch steigt der Meeresspiegel. Verglichen mit diesen Wirkungsketten erscheint ein Zusammenhang zwischen der Abnahme arktischen Seeeises und kälteren Wintern in den Vereinigten Staaten zunächst kontraintuitiv. \citep{wallace-2014}

Mit Änderungen in der globalen Zirkulation ließe sich auch in einem wärmeren Klima, das zwar grundsätzlich die Wahrscheinlichkeit für kalte Winter mindert, das häufigere Auftreten kalter Winter erklären. Globale Zirkulationsmuster zeigen sich in der Position der Jetstreams. Verändert sich die globale Zirkulation auf klimatologischen Skalen, manifestiert sich dies in einer Änderung der Position der Jestreams. Der polare Jetstream der nördlichen Hemisphäre führt Luftmassen über Rossby-Wellen von West nach Ost. Über diese Wellen werden kalte Luftmassen südwärts und umgekehrt warme Luftmassen nordwärts transportiert. So bekommen Regionen, die unter einem nach Süden führenden Jet liegen, wahrscheinlich kälteres Wetter zu spüren, während für Regionen, die unter einem nach Norden führenden Jet liegen, der umgekehrte Fall gilt. Sie erfahren wärmeres  Wetter. \citep{palmer-2014}

In dieser Arbeit werden zwei unabhängige Methoden zur Detektion von Jetstreams und zur Differenzierung zwischen Polarfront- und subtropischem Jetstream auf einen Datensatz angewandt, der aus Monatsmitteln der ERA-Interim- und ERA-40-Daten von 1957 bis 2017 besteht. Die Positionen der Jetstreams sowie deren Zonal- und Meridionalwindgeschwindigkeiten werden daraufhin auf Veränderungen in Zeit und Raum untersucht. Dies geschieht sowohl auf jährlichen als auch auf saisonalen Skalen. Darüber hinaus wird der Zusammenhang zwischen der arktischen Amplifikation, der sich verstärkenden Erwärmung der Arktis, und den Veränderungen des Polarfrontjets analysiert.

%\section*{Gliederung}
Gegliedert ist die Arbeit hierzu in folgende Abschnitte: Aufbauend auf der Motivation wird in Kapitel \ref{ch:hintergrund} der Stand der Forschung präsentiert. Darauffolgend werden in Kapitel \ref{ch:methodik} die genutzten Datensätze sowie die beiden verwendeten Methoden zur Detektion von Jetstreams vorgestellt. In Kapitel \ref{ch:plausibilitaet} wird zunächst anhand von zufällig ausgewählten Einzelfällen überprüft, wie plausibel die gefunden Jetstream\-positionen sind, bevor die beiden Methoden mittels Streudiagrammen miteinander verglichen werden. Die zeitliche Analyse der Ergebnisse findet in Kapitel \ref{ch:klimatologie} statt, in dem Trends der Positionen und Windgeschwindigkeiten ebenso wie die Entwicklung der Variabilität untersucht werden. Darüber hinaus wird mit Hovmöller-Diagrammen die raumzeitliche Entwicklung analysiert. In Kapitel \ref{ch:seeeis} wird untersucht, ob ein Zusammenhang zwischen der Ausdehnung des arktischen Seeeises und der Position des Polarfrontjetstreams und dessen Variabilität besteht. Zum Schluss werden die Ergebnisse zusammengefasst.
