\chapter{Zusammenfassung} \label{ch:diskussion}

In dieser Arbeit wurden zwei verschiedene Methoden der Jetstream-Detektion vorgestellt. Diese weisen systematische Unterschiede auf. So sucht die Chebyshev-Methode nach den zwei intensivsten Maxima des Zonalwinds auf jedem Längengradabschnitt und findet so auch kürzere Jets, während die Dijkstra-Methode (entnommen aus \citep{molnos-2017}) nach kürzesten Pfaden sucht und so globale, einmal den Pol umlaufende Jetstreams lokalisiert. Die Chebyshev-Methode betrachtet lediglich die Zonalwindkomponente, die Dijkstra-Methode nutzt beide Komponenten des Horizontalwindfelds. Beide Methoden sind in der Lage, zwischen Polarfront- und subtropischem Jet zu unterscheiden.

Angewandt wurden beide Methode auf einen Datensatz, der die ERA-Interim-Daten mit der ERA-40-Reanalyse verlängert. Dieser wurde auf die Nordhemisphäre zugeschnitten, auf ein T63-Gitter interpoliert und monatlich gemittelt. Die Überprüfung des Datensatzes auf Homogenität in der Polarregion, den mittleren Breiten und den Tropen zeigt, dass vor allem der Zonalwind keinen Sprung zwischen der Zeitspanne von \num{1957} und \num{1978} und dem Zeitraum ab \num{1979} zeigt. Der Meridionalwind zeigt möglicherweise Änderungen, die jedoch so schwach sind, dass sie für die Jetstream-Detektion als nicht relevant erachtet werden.

Eine Schwäche der Chebyshev-Methode sind Situationen mit mehr als zwei Maxima im Untersuchungsgebiet. Ändert sich dann das Zonalwindfeld stark entlang der Längengradachse, entsteht der Eindruck, es gebe keine(n) homogenen Jetstream(s) sondern lediglich kurze abschnittsweise auftretende Jets. Die deutlichste Schwäche der Dijkstra-Methode, sind hingegen Windfelder, in denen Maxima weit entfernt vom klimatologischen Mittel verlaufen sowie solche in denen kürzere Jets auftreten. So werden Windfelder, die zwar als Jet klassifiziert werden könnten, nicht als solche erkannt, weil die Position zu weit vom Pfad abweicht. Trotz der systematischen Unterschiede der Methoden liegen die Korrelationen für den Polarfront-Jet zwischen \num{0.217} und \num{0.497} und für den subtropischen Jetstream zwischen \num{0.287} und \num{0.571}. Generell sind die Korrelationen für beide Jetstreamklassen von Juni bis November höher als zwischen Dezember und Mai.

Die langzeitlichen Entwicklungen der Mittelwerte für den Polarfrontjet zeigen auf den ersten Blick die systematischen Unterschiede der Methoden, die Chebyshev-Methode modelliert die Position des Jets \SIrange{2}{3}{\degree} weiter nördlich, den Zonalwind etwa \SI{2}{\metre\per\second} schwächer und den Meridionalwind circa \SI{0.5}{\metre\per\second} stärker. Die Analyse zeigt für den Polarfrontjet einen leicht nordwärts wandernden Jetstream nach Chebyshev, nach Dijkstra verlagert er sich leicht südwärts. Die Zonalwindgeschwindigkeit bleibt nach beiden Methoden etwa konstant, während die Meridionalgeschwindigkeit nach Chebyshev schwach und nach Dijkstra deutlich steigt. Für die Zeitreihe der Variabilität oder der Standardabweichungen zeigen die Positionen erneut divergente Ergebnisse. Nach der Chebyshev-Methode steigt sie, während sie nach Dijkstra einen Abwärtstrend aufweist. Die Variabilität des Zonalstroms nimmt übereinstimmend zu und die Variabilität der Meridionalkomponente übereinstimmend ab.

Auch für den Subtropenjet zeigen sich systematische Unterschiede, der Jet verläuft nach Chebyshev ungefähr \SI{2}{\degree} weiter nördlich, der Zonalwind ist nach Dijkstra etwa \SIrange{2}{4}{\metre\per\second} stärker und der Meridionalwind wiederum nach Chebyshev circa \SI{0.5}{\metre\per\second} stärker. Der Trend der Positionen ist erneut divergent. Während die Chebyshev-Methode einen nordwärts gerichteten Trend erkennt, ist der Trend für die Dijkstra-Methode schwach südwärts gerichtet. Für den Zonalwind ist der Trend übereinstimmend positiv. Ebenso wird der Trend des Meridionalwinds für beide Methoden übereinstimmend positiv modelliert. Die Zeitreihe der Variabilität der Positionen zeigt einen schwach negativen bis nicht existenten Trend. Für die Zeitreihen der Standardabweichung der Zonal- und Meridionalwindkomponente gibt es die größte Übereinstimmung. Die Trends liegen sehr nahe beieinander und zeigen einen Positivtrend für die Variabilität des Zonalstroms sowie einen leichten negativen Trend für die Meridionalkomponente.

Bei der Analyse der Korrelationen zwischen der arktischen Meereisbedeckung und der gefunden Jetpositionen zeigen beide Methoden übereinstimmend signifikante Ergebnisse. Erstens gibt es eine schwache Korrelation zwischen abnehmender Seeeisausdehnung und den Positionen des Polarfrontjets, das heißt bei niedrigerer Meereisausdehnung liegt der Polarfrontjet eher nördlich. Zweitens ist die Korrelation zwischen abnehmender Meereisfläche und der Variabiliät der Positionen sehr schwach und besagt, dass bei abnehmender Meereisfläche auch die Variabilität der Positionen abnimmt.
