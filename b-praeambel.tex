%% Präambel für Master Thesis
%  Dokumentenklasse
\documentclass[a4paper,11pt,twoside,onecolumn,openright,final]{memoir} %scrartcl scrreprt scrbook %memoir

%% Pakete
% Unterscheidung zwischen compiler LuaLaTeX vs pdfLaTeX
\usepackage{ifluatex}
\ifluatex 
    % lualatex
    %\usepackage{polyglossia}   
    \usepackage{fontspec}
    %\defaultfontfeatures{Ligatures=TeX}
    %\usepackage[]{unicode-math} 
    %\unimathsetup{math-style=TeX}
    \setmainfont[Ligatures=TeX]{DejaVuSerif} 
    \setsansfont{DejaVuSans}
    \setmonofont{DejaVuSansMono}
\else 
    % pdflatex
    \usepackage[T1]{fontenc}
    \usepackage[utf8]{inputenc}
    % Schriftarten
    \usepackage{palatino}       % Serifen %times
    \usepackage[scaled]{helvet} % Serifenlos
    \usepackage{courier}        % Schreibmaschinenschrift
    %\usepackage{mathptmx}      % Matheschrift
\fi

%% Sprache
\usepackage[ngerman, british]{babel}


%% Allgemeines Layout
\setbinding{.5cm} % bindekorrektur
\semiisopage % \semiisopage[12] \isopage
\OnehalfSpacing % zeilenabstand
%\usepackage[a4paper, top=40mm, inner=30mm, outer=25mm, bottom=30mm]{geometry}
\usepackage[document]{ragged2e} % linksbündig 
\usepackage[babel,german=quotes]{csquotes} % Anführungszeichen
\usepackage{color}                    % farbe 

%% Layout Kopf- und Fußzeile
% plain
\makeevenhead{plain}{}{}{}
\makeevenfoot{plain}{}{\thepage}{}
\makeoddhead{plain}{}{}{}
\makeoddfoot{plain}{}{\thepage}{}
% headings
\makeevenhead{headings}{\leftmark}{}{}
\makeevenfoot{headings}{}{\thepage}{}
\makeoddhead{headings}{}{}{\rightmark}
\makeoddfoot{headings}{}{\thepage}{}
% sep line
%\makeheadrule{plain}{\textwidth}{.5pt}
\makefootrule{plain}{\textwidth}{.5pt}{0ex}
\makeheadrule{headings}{\textwidth}{.5pt}
\makefootrule{headings}{\textwidth}{.5pt}{0ex}

%% Querverweise
\usepackage[linktoc=all, hypertexnames=false, plainpages=false, pdfpagelabels]{hyperref}
\usepackage{memhfixc}

%% Inhaltsverzeichnis
%\usepackage[nottoc,numbib]{tocbibind}
\setcounter{tocdepth}{1}

%% Sonstige Pakete
%% Matheumgebungen
\usepackage[fleqn]{mathtools}

%% Abbildungen
\usepackage{tikz}
\usepackage{graphicx}
\usepackage{float} % für Fließumgebungen; Platzierung H verschiebt nicht
\restylefloat{figure}
\ifluatex
    \newcommand{\includegraphicstikz}{\input}
\else
    \newcommand{\includegraphicstikz}{\includegraphics}
\fi

%% Captions von Tabellen und Abbildungen
\makeatletter
\renewcommand{\fnum@table}[1]{\textbf{\tablename~\thetable:} }
\renewcommand{\fnum@figure}[1]{\textbf{\figurename~\thefigure:} }
\makeatother

%% Tabellen
\usepackage{tabularx}
\usepackage{multirow}

%% Zitation und Literaturverzeichnis
\usepackage[square, semicolon]{natbib}
\bibliographystyle{abbrvnat} % plainnat abbrvnat unsrtnat
\usepackage{url}


%% Blindtexte
\usepackage{blindtext, lipsum}








%\usepackage{graphics, graphicx, caption, array, longtable}
%\usepackage{marginnote, lmodern, enumitem}
%\usepackage{tabularx, color, url, pdflscape, subfigure}
%\usepackage{siunitx, amsfonts, amsmath, amsthm, amsbsy, amssymb}
%\usepackage[e]{esvect} % bessere Vektorpfeile

% %\usepackage{chicago}
% \usepackage[fixlanguage]{babelbib}
% \selectbiblanguage{ngerman}
% 
%
% \usepackage{lipsum}
% \usepackage{array}
% %


%
%% ## Ende der Präambel
