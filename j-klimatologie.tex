\chapter[Zeitliche und räumliche Entwicklung]{Zeitliche und räumliche Entwicklung der Jetstreams}
\label{ch:klimatologie}

In diesem Kapitel werden die gefundenen Jetstream-Positionen und die damit verbundenen Windgeschwindigkeiten auf ihre zeitliche Entwicklung untersucht. Hierfür werden einerseits jährliche räumliche Mittelwerte und Standardabweichungen der Breitengradpositionen, der Zonal- und der Meridionalwindkomponenten nach beiden Methoden als Zeitreihe dargestellt sowie eine lineare Regression angewandt. Die raumzeitliche Analyse basiert auf dem Trog-und-Rücken-Diagramm von \citet{hovmoeller-1949}. Deren ursprünglicher Zweck ist eine vereinfachte Darstellung der Wellenbewegungen der oberen Atmosphäre gewesen. Hierfür sind auf der Abszisse die Längengrade und auf der Ordinate die Zeit aufgetragen worden, während auf der Fläche mittels Konturen das Geopotenzial abgebildet wurde. Auf diese Weise sind neben Mustern der globalen Zirkulation auch die Verlagerungsgeschwindigkeit von planetaren Wellen ermittelt worden. In dieser Arbeit werden anstelle des Geopotenzials die Positionen der Jetstreams sowie die Windkomponenten untersucht, um deren raumzeitliche Entwicklung zu untersuchen, wobei der Fokus an dieser Stelle auf der Chebyshev-Methode liegt.


\section{Entwicklung des meridionalen Mittels}

Um die zeitliche Entwicklung zu untersuchen, werden die Jetstream-Positionen und deren zugehörige Windgeschwindigkeiten komponentenweise jährlich über alle Meridiane gemittelt. Ebenso wird vorgegangen, um die zeitliche Entwicklung der Standardabweichungen zu bestimmen. Der Vergleichbarkeit halber werden beide Methoden visualisiert. Aufgrund der unterschiedlichen Systematik der Methoden sind keineswegs gleiche Ergebnisse zu erwarten. Ähnliche Tendenzen sind aber durchaus wünschenswert. %Darüber hinaus sind die Ergebnisse in Tabellenform in Anhang \ref{ch:asd} auch saisonal abgebildet.

In den Abbildungen \ref{fig:pfj-mean} und \ref{fig:pfj-sd} ist die Entwicklung der Mittelwerte und Standardabweichungen für den Polarfrontjet von 1957 bis 2017 dargestellt. Dieser liegt nach Chebyshev bei einem Mittelwert von \SI{62.58}{\degree} und einer Standardabweichung von \SI{11.09}{\degree}, während Dijkstra ihn etwas weiter südlich modelliert. Hier liegt er bei einem Mittelwert von \SI{59.57}{\degree} und einer Standardabweichung von \SI{7.90}{\degree}. Die Standardabweichung liegt erwartungsgemäß niedriger, weil die Chebyshev-Methode eine größere Schwankung erlaubt. Der Zonalwind des Polarfront-Jetstreams liegt bei einem Mittelwert von \SI{17.27}{\metre\per\second} und einer Standardabweichung von \SI{8.46}{\metre\per\second}. Nach der Dijkstra-Methode sind diese mit einem Mittelwert von \SI{19.43}{\metre\per\second} und einer Standardabweichung von \SI{7.42}{\metre\per\second} intensiver und unterliegen einer geringeren Schwankung. Die Meridionalwindkomponente liegt nach Chebyshev bei einem Mittelwert von \SI{-0.54}{\metre\per\second} und einer Standardabweichung von \SI{6.75}{\metre\per\second}. Die Dijkstra-Methode zeigt hier einen Mittelwert von \SI{-0.98}{\metre\per\second} und eine Standardabweichung von \SI{8.31}{\metre\per\second}, was einer stärker südwärts gerichteten Windkomponente mit stärkerer Schwankung enstpricht. Dies verdeutlicht einen weiteren Unterschied der beiden Modelle. Während die Dijkstra-Methode den Betrag der horrozontalen Windgeschwindigkeit und die Windrichtung einbezieht, stützt sich die Chebyshev-Methode einzig auf die Zonalwindkomponente. So besteht die Möglichkeit, dass südwärts oder nordwärts gerichtete Jets nicht erfasst werden, falls an diesen Stellen nicht ebenfalls ein Maximum des Zonalwinds detektiert wird.

Lineare Trends werden von beiden Methoden nur teilweise übereinstimmend erkannt. So zeigt der Mittelwert der Positionen nach Chebyshev einen schwach positiven Trend, was einer Verlagerung nach Norden entspricht, während die Dijkstra-Methode eine ähnlich schwache Tendenz mit umgekehrtem Vorzeichen zeigt. Der Trend der Mittelwerte des Zonalwinds ist übereinstimmend nicht signifikant. Für die Mittelwerte des Meridionalwinds zeigen die Daten einen klaren positiven Trend, was bedeutet, dass sich der Meridionalwind näher an $0$ annähert. Hier ist der lineare Trend für die Windkomponenten nach Dijkstra wesentlich stärker als nach der Chebyshev-Methode. Die Standardwabweichungen der Jetstream-Positionen zeigen für die Chebyshev-Methode einen positiven Trend, während dieser für die Dijkstra-Methode schwach negativ ausfällt. Für die Standardabweichungen der Zonalwindkomponente sind die Tendenzen für beide Methoden übereinstimmend positiv, für die Meridionalwindkomponente hingegen übereinstimmend negativ.

In den Abbildungen \ref{fig:stj-mean} und \ref{fig:stj-sd} ist die Entwicklung für den subtropischen Jetstream dargestellt. Dieser liegt nach Chebyshev bei einem Mittelwert von \SI{33.97}{\degree} und einer Standardabweichung von \SI{8.52}{\degree}, während Dijkstra ihn weiter südlich modelliert. Hier liegt er bei einem Mittelwert von \SI{31.42}{\degree} und einer Standardabweichung von \SI{6.54}{\degree}. Die Standardabweichung liegt auch beim Subtropen-Jetstream erwartungsgemäß niedriger, weil die Chebyshev-Methode eine größere Schwankung erlaubt. Der Zonalwind des Polarfront-Jetstreams erreicht Geschwindigkeiten um einen Mittelwert von \SI{33.97}{\metre\per\second} und eine Standardabweichung von \SI{13.58}{\metre\per\second}. Nach der Dijkstra-Methode sind diese mit einem Mittelwert von \SI{34.83}{\metre\per\second} und einer Standardabweichung von \SI{13.74}{\metre\per\second} intensiver. Die Schwankung ist ähnlich. Die Meridionalwindkomponente liegt nach Chebyshev bei einem Mittelwert von \SI{1.92}{\metre\per\second} und einer Standardabweichung von \SI{5.82}{\metre\per\second}. Die Dijkstra-Methode zeigt hier einen Mittelwert von \SI{1.58}{\metre\per\second} und eine Standardabweichung von \SI{6.04}{\metre\per\second}, was eine schwächeren Nordkomponente bei ähnlicher Standardabweichung zeigt.

Lineare Trends werden auch beim Subtropenjet von beiden Methoden nur teilweise übereinstimmend erkannt. So zeigt der Mittelwert der Positionen nach Chebyshev mit einem positiven Trend eine nördlich gerichtete Verlagerung, während die Dijkstra-Methode eine schwache negative Tendenz zeigt. Der Trend der Mittelwerte des Zonalwinds ist übereinstimmend positiv, wobei die Dijkstra-Methode den stärkeren Anstieg ermittelt. Für die Mittelwerte des Meridionalwinds sind die Trends nach Chebyshev schwach positiv und nach Dijkstra deutlich positiv, was eine stärkere nördliche Komponente der Windgeschwindigkeit im Jetstream zeigt. Für die Standardabweichungen der Positionen ist der Trend nach Chebyshev nicht signifikant, während die Dijkstra-Methode eine schwach südwärts gerichte Tendenz feststellt. Für die Standardabweichungen des Zonalwinds zeigen beide Methoden übereinstimmend einen deutlich positiven Trend. Die Übereinstimmung der Methoden in diesem Fall ist außergewöhnlich groß, wie sich sowohl in den Absolutwerten vor allem aber im linearen Trend zeigt. Diese überlagern einander innerhalb des 95\%-Konfidenzintervalls. Auch für den Trend der Meridionalwindkomponente gibt es eine Übereinstimmung, beide Methoden zeigen einen negativen Trend, was einer Abschwächung der nordwärts gerichteten Windkomponente entspricht.


\begin{figure} %% Klimatrend PFJ MEAN
  \centering
  \begin{minipage}{\textwidth}
    \centering
      \includegraphicstikz{../05-visu-pdf-tikz/04-clim/pfj-lat-mn}
  \end{minipage} \\ 
    \begin{minipage}{\textwidth}
      \centering
      \includegraphicstikz{../05-visu-pdf-tikz/04-clim/pfj-u-mn}
  \end{minipage} \\ 
  \begin{minipage}{\textwidth}
    \centering
      \includegraphicstikz{../05-visu-pdf-tikz/04-clim/pfj-v-mn}
  \end{minipage}
  \caption[Zeitreihe der Mittelwerte der Positionen, des Zonal- und Meridionalwinds des Polarfrontjets nach Chebyshev und Dijkstra]{Zeitreihe der jährlichen Mittelwerte der Breitengradpositionen (oben), Zonal- (mittig) und Meridionalwindgeschwindigkeiten (unten) und Trend des Polarfrontjetstreams nach Chebyshev- (rot) und Dijkstra-Methode (blau) im Vergleich} \label{fig:pfj-mean}
\end{figure}

\begin{figure} %% Klimatrend PFJ SD
  \centering
  \begin{minipage}{\textwidth}
    \centering
      \includegraphicstikz{../05-visu-pdf-tikz/04-clim/pfj-lat-sd}
  \end{minipage} \\ 
    \begin{minipage}{\textwidth}
      \centering
      \includegraphicstikz{../05-visu-pdf-tikz/04-clim/pfj-u-sd}
  \end{minipage} \\ 
  \begin{minipage}{\textwidth}
    \centering
      \includegraphicstikz{../05-visu-pdf-tikz/04-clim/pfj-v-sd}
  \end{minipage}
  \caption[Zeitreihe der Standardabweichungen der Positionen, des Zonal- und Meridionalwinds des Polarfrontjets nach Chebyshev und Dijkstra]{Zeitreihe der jährlichen Standardabweichungen der Breitengradpositionen (oben), Zonal- (mittig) und Meridionalwindgeschwindigkeiten (unten) und Trend des Polarfrontjetstreams nach Chebyshev- (rot) und Dijkstra-Methode (blau) im Vergleich} \label{fig:pfj-sd}
\end{figure}

\begin{figure} %% Klimatrend STJ MEAN
  \centering
  \begin{minipage}{\textwidth}
    \centering
      \includegraphicstikz{../05-visu-pdf-tikz/04-clim/stj-lat-mn}
  \end{minipage} \\ 
    \begin{minipage}{\textwidth}
      \centering
      \includegraphicstikz{../05-visu-pdf-tikz/04-clim/stj-u-mn}
  \end{minipage} \\ 
  \begin{minipage}{\textwidth}
    \centering
      \includegraphicstikz{../05-visu-pdf-tikz/04-clim/stj-v-mn}
  \end{minipage}
  \caption[Zeitreihe der Mittelwerte der Positionen, des Zonal- und Meridionalwinds des Subtropenjets nach Chebyshev und Dijkstra]{Zeitreihe der jährlichen Mittelwerte der Breitengradpositionen (oben), Zonal- (mittig) und Meridionalwindgeschwindigkeiten (unten) und Trend des subtropischen Jetstreams nach Chebyshev- (rot) und Dijkstra-Methode (blau) im Vergleich} \label{fig:stj-mean}
\end{figure}

\begin{figure} %% Klimatrend STJ SD
  \centering
  \begin{minipage}{\textwidth}
    \centering
      \includegraphicstikz{../05-visu-pdf-tikz/04-clim/stj-lat-sd}
  \end{minipage} \\ 
    \begin{minipage}{\textwidth}
      \centering
      \includegraphicstikz{../05-visu-pdf-tikz/04-clim/stj-u-sd}
  \end{minipage} \\ 
  \begin{minipage}{\textwidth}
    \centering
      \includegraphicstikz{../05-visu-pdf-tikz/04-clim/stj-v-sd}
  \end{minipage}
  \caption[Zeitreihe der Standardabweichungen der Positionen, des Zonal- und Meridionalwinds des Subtropenjets nach Chebyshev und Dijkstra]{Zeitreihe der jährlichen Standardabweichungen der Breitengradpositionen (oben), Zonal- (mittig) und Meridionalwindgeschwindigkeiten (unten) und Trend des subtropischen Jetstreams nach Chebyshev- (rot) und Dijkstra-Methode (blau) im Vergleich} \label{fig:stj-sd}
\end{figure}

\section{Globale Entwicklung des Polarfront-Jetstreams}
Um die langfristigen Änderungen der Jetstreams über den mittleren Breiten herauszuarbeiten, wird der Fokus auf den Polarfrontjet und die Cheby\-shev-Methode gelegt, da diese eine gößere Variabilität erlaubt. Für die genutzten Hovmöller-Diagramme werden die Jetstream-Positionen ebenso wie die horizontalen Windkomponenten zunächst saisonal gruppiert. Auf diese saisonalen Daten wird ein fünfjähriges gleitendes Mittel angewandt, sodass für jede der untersuchten Größen Breitengrad, Zonalwind und Meridionalwind vier saisonale Hovmöller-Diagramme entstehen. Der Vollständigkeit halber finden sich die Hovmöller-Diagramme für den subtropischen Jetstream sowie für die Dijkstra-Methode in Anhang \ref{ch:app-hovm}.

In Abbildung \ref{fig:hovm-pfj-lat-warm} oben sind besagte Hovmöller-Diagramme für Frühling und Sommer dargestellt. Im Frühjahr verläuft der Polarfrontjet über dem Pazifik relativ weit nördlich zwischen \SI{75}{\degree} und \SI{60}{\degree}. Mit dem Auftreffen auf die Kontinentalfläche Nordamerikas und deren Gebirgsketten an der Westküste weicht der Jet auf Bereiche zwischen \SI{50}{\degree} und \SI{65}{\degree} aus. Über dem Atlantik wiederum streut die Position recht stark und liegt zwischen \SI{50}{\degree} und \SI{70}{\degree}. Der Europäische Kontinent drängt ihn wieder auf leicht nördlichere Routen zwischen \SI{60}{\degree} und \SI{70}{\degree}. Der Himalaya lenkt den Polarfrontjet südlich auf etwa \SI{45}{\degree} ab. Rückseitig des Himalayas verlagert er sich nördliche, bevor er über dem Pazifik zwischen \SI{60}{\degree} und \SI{75}{\degree} verläuft.

In der zeitlichen Entwicklung sind Schwankungen des Jetstreams nach Nord und Süd erkennbar. Besonders variabel ist der Jet über dem Atlantik, besonders konstant über dem Himalaya. Über dem Pazifik zeichnet sich eine schwache Tendenz zu südlicher verlaufenden Jets ab, nachdem 1975 mit etwa \SI{75}{\degree} das Maximum markiert. Über der Ostküste Nordamerikas ist möglicherweise ein wiederkehrendes Muster erkennbar. Der Jet oszilliert dort auf Zeitskalen von 20\,Jahren zwischen unter \SI{50}{\degree} und bis zu \SI{65}{\degree}.

In Abbildung \ref{fig:hovm-pfj-lat-warm} unten ist das Hovmöller-Diagramm für die Jetstream-Position für den Sommer dargestellt. Hier verläuft der Jet insgesamt weiter nördlich zwischen \SI{55}{\degree} und \SI{75}{\degree}. Vom Pazifik kommend verlagert er sich über den Gebirgsketten an der Westküste Nordamerikas südlich auf etwa \SI{60}{\degree}, bevor er sich rückseitig nahe des Atlantiks leicht nördlich verlagert auf \SI{65}{\degree} bis \SI{70}{\degree}. Beim Auftreffen auf den Kontinent verlagert er sich wieder südwärts auf \SI{55}{\degree} bis \SI{60}{\degree}. Über Asien ist der Verlauf weit nördlich zwischen \SI{65}{\degree} und \SI{75}{\degree}, sodass das markante Himalaya-Gebirge nicht wie in Abbildung \ref{fig:hovm-pfj-lat-warm} oben erkennbar ist. Erst beim Übergang Asien/Pazifik ändert sich an dieser Position etwas, er verlagert sich leicht nordwärts auf bis zu \SI{75}{\degree}.

Im zeitlichen Verlauf sind Variabilitäten zwischen \SI{55}{\degree} und \SI{65}{\degree} an der Westküste Nordamerikas zu sehen, diese scheinen auf Skalen von $10$ bis $15$\,Jahren wiederzukehren. Weitere sich zeitlich wiederholende Muster sind nicht erkennbar.

In Abbildung \ref{fig:hovm-pfj-lat-cold} oben ist das Hovmöller-Diagramm für Herbst dargestellt. Er verläuft global wieder weiter südlich zwischen \SI{50}{\degree} und \SI{75}{\degree}. Vom Pazifik bei etwa \SI{70}{\degree} auf das nordamerikanische Festland treffend verläuft er dort weiter südlich zwischen \SI{50}{\degree} und \SI{65}{\degree}. Über dem Festland verläuft er annähernd zonal, bevor er sich am Übergang von Nordamerika zum Nordatlantik leicht nördlich auf bis zu \SI{70}{\degree} verlagert. Beim Auftreffen auf den eurasischen Kontinent findet eine südliche Verlagerung statt. Über Europa liegt er zwischen \SI{60}{\degree} und \SI{70}{\degree}. Über dem Himalaya verlagert er sich Richtung Süden auf \SI{50}{\degree} bis \SI{60}{\degree}, bevor er am Übergang von Asien zum Pazifik auf nördlichere Pfade zwischen \SI{65}{\degree} und \SI{75}{\degree} ausweicht.

Zeitlich wiederkehrende Muster sind in Ansätzen erkennbar am Übergang vom Pazifik zu Nordamerika mit besonders südlich verlaufenden Jets um die Jahre 1975, 1982, 1995 und 2005. Auch am Übergang vom Atlantik zu Europa sind südlicher verlaufende Jets in den Jahren 1962, 1978 und 2010 erkennbar. Von 1985 bis 1990 verläuft der Polarfrontjet weiter nördlich als üblich bei über \SI{70}{\degree}. Über dem Himalaya scheint er auf Zeitskalen von 10 bis 15\,Jahren zwischen eher nördlich verlaufenden Routen um \SI{60}{\degree} und eher südlich Verlaufenden um \SI{50}{\degree} zu oszillieren.

In Abbildung \ref{fig:hovm-pfj-lat-cold} unten ist das Hovmöller-Diagramm für den Polarfrontjet im Winter dargestellt. Über dem Pazifik verläuft er weit nördlich bei etwa \SI{70}{\degree}. Über dem Westen Nordamerikas wird er nach Süden auf Pfade zwischen \SI{50}{\degree} und \SI{60}{\degree} abgelenkt. Über dem Nordatlantik verläuft er wieder weiter nördlich zwischen \SI{65}{\degree} und \SI{75}{\degree}, bevor er am Übergang zum europäischen Festland wieder südlich auf etwa \SI{60}{\degree} gedrängt wird. Die markanteste Ablenkung Richtung Süden findet am Himalaya statt. Dort verläuft er zwischen \SI{50}{\degree} und \SI{55}{\degree}. Ab dem Übergang zum Pazifik ist sein Verlauf wieder nördlicher bei \SI{70}{\degree}.

In den Wintermonaten zeigen sich wiederkehrende Muster auf dem Pazifik, auf dem Atlantik, über Mitteleuropa und über Asien. Auf dem Pazifik verläuft er in Abständen von 10 bis 15\,Jahren eher nördlich bzw. südlich. Auf dem Atlantik ist die Ausprägung schwächer, dennoch ist eine Variabilität auf Zeitskalen von 10 bis 20\,Jahren erkennbar, wobei vor allem das letzte Maximum um 2015 besonders schwach ausfällt. Von \SI{0}{\degree} bis \SI{60}{\degree} geographischer Länge zeigt sich eine Oszillation auf Skalen von 10 bis 20\,Jahren, die über diesen Korridor konstant ausfällt. Das heißt, verläuft der Polarfrontjet eher auf der Breite von Mitteleuropa, verbleibt er bei dieser Breite dort bis zum Himalaya. Trifft er den europäischen Kontinent jedoch auf der Breite von Skandinavien, verläuft er von dort weiterhin nördlich. Über dem Himalaya wechseln sich eher nördliche Pfade mit eher Südlichen innerhalb von ca.\,15\,Jahren ab.

\subsection{Polarfront-Jetstream}

\begin{figure} %% Hovmöller PFJ LAT
  \centering
  \begin{minipage}{\textwidth}
    \centering
      \includegraphicstikz{../05-visu-pdf-tikz/05-hovm-lat/hovm-chebyshev-pfj-mam}
  \end{minipage} \\ 
    \begin{minipage}{\textwidth}
      \centering
      \includegraphicstikz{../05-visu-pdf-tikz/05-hovm-lat/hovm-chebyshev-pfj-jja}
  \end{minipage} \\ 
  \caption[Hovmöllerdiagramme der Positionen des Polarfrontjets nach Chebyshev im Frühjahr und Sommer]{Hovmöller-Diagramme für die Breitengradpositionen des Polarfrontjetstreams nach der Chebyshev-Methode im Frühjahr (oben) und Sommer (unten)} \label{fig:hovm-pfj-lat-warm}
\end{figure}

\begin{figure}
  \centering
  \begin{minipage}{\textwidth}
    \centering
      \includegraphicstikz{../05-visu-pdf-tikz/05-hovm-lat/hovm-chebyshev-pfj-son}
  \end{minipage} \\ 
    \begin{minipage}{\textwidth}
      \centering
      \includegraphicstikz{../05-visu-pdf-tikz/05-hovm-lat/hovm-chebyshev-pfj-djf}
  \end{minipage} \\ 
  \caption[Hovmöllerdiagramme der Positionen des Polarfrontjets nach Chebyshev im Herbst und Winter]{Hovmöller-Diagramme für die Breitengradpositionen des Polarfrontjetstreams nach der Chebyshev-Methode im Herbst (oben) und Winter (unten)} \label{fig:hovm-pfj-lat-cold}
\end{figure}

In den Abbildungen \ref{fig:hovm-pfj-u-warm} und \ref{fig:hovm-pfj-u-cold} sind die räumlichen und zeitlichen Veränderungen der Zonalwindgeschwindigkeiten für Frühling, Sommer, Herbst und Winter dargestellt. Im Frühling in Abbildung \ref{fig:hovm-pfj-u-warm} oben ist der Jet mit \SI{0}{\metre\per\second} bis \SI{20}{\metre\per\second} verhältnismäßig schwach. Über dem nordamerikanischen Kontinent beschleunigt er auf \SI{15}{\metre\per\second} bis \SI{25}{\metre\per\second}, bevor er über dem Atlantik mit \SI{10}{\metre\per\second} bis \SI{20}{\metre\per\second} wieder abschwächt. Mit dem Auftreffen auf Europa steigt seine Geschwindidkeit wieder auf \SI{15}{\metre\per\second} bis \SI{25}{\metre\per\second}. Auf eine schwächere Phase mit \SI{10}{\metre\per\second} bis \SI{20}{\metre\per\second} folgen über dem Himalaya hohe Windgeschwindigkeiten mit über \SI{30}{\metre\per\second}. Rückseitig des Himalayas und auf dem Pazifik schwächen sich diese auf \SI{5}{\metre\per\second} bis \SI{15}{\metre\per\second} ab.

Die Zonalwindgeschwindigkeiten im Sommer sind in Abbildung \ref{fig:hovm-pfj-u-warm} unten dargestellt. Sie sind global etwas schwächer und liegen zwischen \SI{5}{\metre\per\second} und \SI{30}{\metre\per\second}. Sie verlaufen über dem Pazifik mit etwa \SI{10}{\metre\per\second} recht schwach, beschleunigen über Nordamerika auf rund \SI{20}{\metre\per\second}. Über dem Atlantik ist der Jet mit \SI{5}{\metre\per\second} bis \SI{15}{\metre\per\second} wieder schwächer. Über dem Nordostatlantik und Westeuropa ist der Zonalwind des Jets mit über \SI{25}{\metre\per\second} am intensivsten. Danach verläuft er weiterhin nördlich am Himalaya und bleibt mit Windgeschwindigkeiten zwischen \SI{10}{\metre\per\second} und \SI{20}{\metre\per\second} schwach bis moderat. Hieran ändert auch der Übergang von Asien zum Pazifik nichts.

In Abbildung \ref{fig:hovm-pfj-u-cold} oben ist der Zonalwind des Polarfrontjets für die Monate September, Oktober und November abgebildet. Er verläuft schwach über dem Atlantik mit etwa \SI{10}{\metre\per\second}, mit \SIrange{20}{30}{\metre\per\second} über Nordamerika stärker. Am Übergang von Nordamerika zum Atlantik schwächt der Jetstream sich auf \SI{10}{\metre\per\second} bis \SI{20}{\metre\per\second} ab. Über Europa legt seine Geschwindigkeit wieder auf über \SI{20}{\metre\per\second} bis \SI{30}{\metre\per\second} zu, auf das eine Phase leicht schwächerer Zonalwinde mit unter \SI{20}{\metre\per\second}. Über dem Himalaya beschleunigt er auf Geschwindigkeiten zwischen \SI{20}{\metre\per\second} und \SI{25}{\metre\per\second}. Die Zonalwindgeschwindigkeiten sind über dem Pazifik schwach und liegen bei \SIrange{5}{15}{\metre\per\second}.


Die Zonalwindgeschwindigkeiten des Polarfront-Jetstreams für die Wintermonate sind in Abbildung \ref{fig:hovm-pfj-u-cold} unten dargestellt. Sie liegen global zwischen \SI{0}{\metre\per\second} und \SI{30}{\metre\per\second}. Der Jet zeichnet sich hier über dem Pazifik durch niedrige Windgeschwindigkeiten aus, bevor er mit dem Auftreffen auf den nordamerikanischen Kontinent auf \SI{20}{\metre\per\second} bis \SI{30}{\metre\per\second} beschleunigt. Über dem Atlantik schwächt er sich auf \SI{0}{\metre\per\second} bis \SI{15}{\metre\per\second} ab. Über dem Ostatlantik und Westeuropa liegen die Geschwindigkeiten hoch zwischen \SI{20}{\metre\per\second} und \SI{30}{\metre\per\second}. Über dem Westen Russlands schwächt er sich wie auch in Frühjahr und Herbst ab, bevor er über dem Himalaya auf etwa \SI{25}{\metre\per\second} zulegt. Über dem Pazifik sind die Zonalgeschwindigkeiten niedrig mit \SI{0}{\metre\per\second} bis \SI{10}{\metre\per\second}.



\begin{figure} %% Hovmöller PFJ U
  \centering
  \begin{minipage}{\textwidth}
    \centering
      \includegraphicstikz{../05-visu-pdf-tikz/06-hovm-u/hovm-chebyshev-pfj-mam}
  \end{minipage} \\ 
    \begin{minipage}{\textwidth}
      \centering
      \includegraphicstikz{../05-visu-pdf-tikz/06-hovm-u/hovm-chebyshev-pfj-jja}
  \end{minipage} \\ 
  \caption[Hovmöllerdiagramme des Zonalwinds des Polarfrontjets nach Chebyshev im Frühjahr und Sommer]{Hovmöller-Diagramme für die Zonalwindgeschwindigkeiten des Polarfrontjetstreams nach der Chebyshev-Methode im Frühjahr (oben) und Sommer (unten)} \label{fig:hovm-pfj-u-warm}
\end{figure}

\begin{figure}
  \centering
  \begin{minipage}{\textwidth}
    \centering
      \includegraphicstikz{../05-visu-pdf-tikz/06-hovm-u/hovm-chebyshev-pfj-son}
  \end{minipage} \\ 
    \begin{minipage}{\textwidth}
      \centering
      \includegraphicstikz{../05-visu-pdf-tikz/06-hovm-u/hovm-chebyshev-pfj-djf}
  \end{minipage} \\ 
  \caption[Hovmöllerdiagramme des Zonalwinds des Polarfrontjets nach Chebyshev im Herbst und Winter]{Hovmöller-Diagramme für die Zonalwindgeschwindigkeiten des Polarfrontjetstreams nach der Chebyshev-Methode im Herbst (oben) und Winter (unten)} \label{fig:hovm-pfj-u-cold}
\end{figure}

In den Abbildungen \ref{fig:hovm-pfj-v-warm} und \ref{fig:hovm-pfj-v-cold} sind die Meridionalwindkomponenten des Polarfrontjets für Frühling, Sommer, Herbst und Winter dargestellt. In den Monaten März, April und Mai ist die Meridionalkomponente des Jets (Abbildung \ref{fig:hovm-pfj-v-warm} oben) über dem Pazifik nordwärts gerichtet, über Nordamerika südwärts, über dem Atlantik wieder nordwärts, über Europa mit viel Varianz nahe $0$, über Mittelasien südwärts und über dem Pazifik wiederum nordwärts. Die Windgeschwindigkeit schwankt zwischen \SI{-10}{\metre\per\second} und \SI{10}{\metre\per\second}

In Abbildung \ref{fig:hovm-pfj-v-warm} ist die zeitliche Entwicklung des Meridionalwindes des Jetstreams dargestellt. Der Meridionalwind ist global etwas schwächer und eher südwärts ausgerichtet. Er liegt zwischen \SI{-8}{\metre\per\second} und \SI{6}{\metre\per\second}. Über dem Pazifik ist er nordwärts ausgerichtet, über dem nordamerikanischen Festland südwärts, über dem Atlantik nordwärts, am Übergang zu Europa indifferent mit hoher Variabilität, über Zentraleuropa vorwiegend nordwärts, über Zentralasien vorderseitig der Hochgebirge und -Plateaus südwärts, rückseitig nordwärts und auf dem Westpazifik südwärts. Da die Intensität des Jetstreams sowohl in der Zonal- als auch in der Meridionalwindkomponente im Sommer am schwächsten ist, sind hier Abweichungen besonders gut erkennbar. So scheint die Intensität der meridionalen Windkomponente zwischen \SI{0}{\degree} bis \SI{45}{\degree} und bei \SI{90}{\degree} bis \SI{105}{\degree} geographischer Länge auf Zeitskalen von etwa $15$\,Jahren zu oszillieren.

In den Monaten September, Oktober und November ist die Spannbreite des Meridionalwinds am größten (siehe Abbildung \ref{fig:hovm-pfj-v-cold} oben). Dieser liegt zwischen \SI{-10}{\metre\per\second} und \SI{15}{\metre\per\second}. Über dem Pazifik ist er nordwärts gerichtet, über Nordamerika südwärts, über dem Atlantik nordwärts, über Europa südwärts, vorderseitig des Altai indifferent bis schwach nordwärts, rückseitig südwärts und über dem Pazifik nordwärts. Ein zeitlicher Trend oder Muster scheinen nicht vorhanden.

Die Wintermonate Dezember, Januar und Februar sind in Abbildung \ref{fig:hovm-pfj-v-cold} unten dargestellt. Die Meridionalwindgeschwindigkeiten liegen zwischen \SI{-15}{\metre\per\second} und \SI{15}{\metre\per\second} und scheinen im Vergleich zu den anderen Jahreszeiten klarer ausgeprägt. So ist die zeitliche Varianz sehr gering. Räumlich verlaufen die Windgeschwindigkeiten über dem Pazifik nordwärts, über Nordamerika weitestgehend südwärts, über dem Atlantik nordwärts, über Europa indifferent bis schwach südwärts, über dem Osten Russlands schwach nordwärts, vorderseitig der asiatischen Hochgebirge südwärts, rückseitig und bis in den Pazifik nordwärts.


\begin{figure} %% Hovmöller PFJ V
  \centering
  \begin{minipage}{\textwidth}
    \centering
      \includegraphicstikz{../05-visu-pdf-tikz/07-hovm-v/hovm-chebyshev-pfj-mam}
  \end{minipage} \\ 
    \begin{minipage}{\textwidth}
      \centering
      \includegraphicstikz{../05-visu-pdf-tikz/07-hovm-v/hovm-chebyshev-pfj-jja}
  \end{minipage} \\ 
  \caption[Hovmöllerdiagramme des Meridionalwinds des Polarfrontjets nach Chebyshev im Frühjahr und Sommer]{Hovmöller-Diagramme für die Meridionalwindgeschwindigkeiten des Polarfrontjetstreams nach der Chebyshev-Methode im Frühjahr (oben) und Sommer (unten)} \label{fig:hovm-pfj-v-warm}
\end{figure}

\begin{figure}
  \centering
  \begin{minipage}{\textwidth}
    \centering
      \includegraphicstikz{../05-visu-pdf-tikz/07-hovm-v/hovm-chebyshev-pfj-son}
  \end{minipage} \\ 
    \begin{minipage}{\textwidth}
      \centering
      \includegraphicstikz{../05-visu-pdf-tikz/07-hovm-v/hovm-chebyshev-pfj-djf}
  \end{minipage} \\ 
  \caption[Hovmöllerdiagramme des Meridionalwinds des Polarfrontjets nach Chebyshev im Herbst und Winter]{Hovmöller-Diagramme für die Meridionalwindgeschwindigkeiten des Polarfrontjetstreams nach der Chebyshev-Methode im Herbst (oben) und Winter (unten)} \label{fig:hovm-pfj-v-cold}
\end{figure}
